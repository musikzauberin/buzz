\documentclass[11pt]{article}
\usepackage{geometry}
\geometry{
 a4paper,
 total={210mm,297mm},
 left=25mm, 
 right=25mm,
 top=25mm,
 bottom=25mm,
 }
 
\usepackage[english]{babel}
\usepackage[utf8x]{inputenc}
\usepackage{amsmath}
\usepackage{graphicx}
\usepackage{multicol}
\usepackage{ragged2e}
\usepackage{tocloft}
\usepackage{gensymb}
\usepackage{siunitx}
\usepackage[hypcap]{caption}
\usepackage{capt-of}
\usepackage{setspace}
\usepackage[parfill]{parskip}
\usepackage{amssymb}
\usepackage{textcomp}
\usepackage{float}
\floatstyle{plain}
\usepackage{color}
\usepackage{epstopdf}
\usepackage{natbib}
\usepackage{hyperref}
\hypersetup{
    colorlinks=true,
    linkcolor=blue,
    filecolor=magenta,      
    urlcolor=cyan,
    citecolor=blue
}
\urlstyle{same}
\usepackage{cleveref}
\usepackage{lastpage}
\usepackage{fancyhdr}
\usepackage{graphicx}
\usepackage[table,xcdraw]{xcolor}
\usepackage[normalem]{ulem}
\useunder{\uline}{\ul}{}

%abstract setup
\renewenvironment{abstract}
 {\hspace{.8cm}
  {\bfseries\huge\abstractname}
  \list{}{
    \setlength{\leftmargin}{.95cm}%
    \setlength{\rightmargin}{\leftmargin}%
  }%
  \item\relax}
 {\endlist}

%random things needed
%\renewcommand{\cftsecleader}{\cftdotfill{\cftdotsep}} %let the content table have dots to number
\linespread{1.6}  %one and half is 1.3, doublespacing is 1.6
\frenchspacing


\begin{document}

\begin{titlepage}
	\centering
	\topskip0pt
	\vspace*{\fill}
	{\huge\bfseries Temporal turnover of plant-pollinator interaction networks \par}
	\vspace{2cm}
	{\Large \textsc{Lim} Jia Le  {    }  CID: 00865029}
	\\ 	\vspace{0.5cm}
	{Department of Biology, Imperial College London, \\Silwood Campus, London, U.K.} \\ \vspace{0.5cm}
	{Submitted in part fulfilment of the requirements for the \\ Bachelor of Science degree in Biology with German for Science \\ at Imperial College London.} \\
	\vspace*{\fill}
	{\large Supervised by\par
	Dr.~Samraat \textsc{Pawar}}
	\vfill
% Bottom of the page
	{\large Last updated: \today\par}
\end{titlepage}

% abstract page
\newpage
\pagenumbering{gobble}
\vspace*{\fill}
\begin{abstract}  
\doublespacing
abstracting abstracting abstracting

\end{abstract}
\vfill

% contents page
\newpage
\pagenumbering{gobble}
\vspace*{\fill}
\tableofcontents 
\vspace*{\fill} 
\thispagestyle{empty}

\doublespacing

% abbreviations page
\newpage 
\vspace*{\fill}
{\huge\bfseries Abbreviations} \\
\\
\\
\large{maybe a table}
\vfill

% Page number
\newpage
\pagestyle{fancy}
\fancyhf{}
\renewcommand{\headrulewidth}{0pt}
\rfoot{Page \thepage \hspace{1pt} of \pageref{LastPage}}
\pagenumbering{arabic}

%%% Start writing
\section{Introduction} % 800 words




\newpage
\section{Materials and Methods} % 1200 words
\subsection{Study area}
The Cerrado spans across most of Central Brazil while extending marginally into Bolivia and Paraguay. It comprises of vegetation ranging from open grasslands to scrublands with a sparse distribution of trees, and smaller regions of gallery and close canopy forests. These patches exist side-by-side, resulting in a highly heterogeneous ecosystem~\citep{Gottsberger2006}. \\
\\
Plant-pollinator interactions were surveyed in the Protected Area of the Jardim Bot\^anico de Bras\'ilia (Bras\'ilia's Botanical Garden Protected Area; hereafter `BBG') and in the Reserva Ecol\'ogica do IBGE (hereafter `IBGE') . The two study sites are located on the Brazilian plateau (1,100m a.s.l.), within the federally protected conservation site ``APA-Gama-Cabe\c ca-de-Viado'', located approximately 30km south of Bras\'ilia (15\degree56'S, 47\degree53'W). This region is characterized by a well-defined wet summer season that lasts from November until March followed by a dry winter period that extends from May until September~\citep{Gottsberger2006a}.\\
\\
The BBG site comprised of a denser type of vegetation with a predominance of large shrubs, lianas, and trees. In contrast, the IBGE site consisted of a 8-hectare plot (200 x 400m) covered with mainly grasses mixed with herbaceous plants and shrubs, with a sparse distribution of lianas and trees.~\citep{Eiten1972}.

\subsection{Sampling methods and species identification}
Bees are the predominant pollinators in the Cerrado ($\sim$70\%) followed by moths ($\sim$12\%), hummingbirds ($\sim$3\%), bats, ($\sim$2\%) and beetles ($\sim$2\%)~\citep{Oliveira2002, Gottsberger2006a, Cappellari2011}. Hence, this study focused only on bee-flower interactions. A plant or pollinator species was included in the surveys only if the flowering plant received visits or if the bee was seen foraging on flowers. For every interaction observed, the plant was tagged with a unique identification number, photographed and vouchered. Plant vouchers were identified by using comparative herbarium material, a checklist of local angiosperms and local botanical expertise (Refer to \hyperref[sec: acknowledgements]{Acknowledgements}).  In both sites, bees were collected with an entomological net and killed either in individual vials with paper pellets moistened with ethyl acetate or frozen after each observation. Insect vouchers were thereafter mounted, preserved, and identified to species level by comparison with reference collections, taxonomic literature, local records~\citep{Moure1962, Silveira2002, Michener2007, Moure2007} and by local entomological experts (Refer to \hyperref[sec: acknowledgements]{Acknowledgements}). \\
\\
The BBG area was sampled weekly (0730h to 1700h) by M. C. Boaventura from June 1995 to June 1997 using two predefined transects (5,280m and 4,130m in length) located 4 km apart~\citep{Boaventura1998}. Sampling in this site totaled 125 days over 25 months (mean = 5 days/month). Interactions involving the introduced honey bee (\textit{Apis mellifera}) were not included in the BBG data set. The IBGE study site was sampled by S.C. Rabeling by walking transects covering the entire area for a full day (0800h to 1700h) at a weekly basis from November 2008 to October 2009. In total, there were 47 sampling days over a 12-month period (mean = 3.91 days/month). 
% why wasnt the honey bee added

\subsection{Climate information}

Data from the IBGE's weather station was used to obtain monthly median temperature and precipitation sum for the past 30 years (1980 - 2010). Median temperature was adopted as monthly distributions of daily average temperatures were skewed. Humidity was not considered as only relative humidity data of IBGE was available.\\
\\
Monthly precipitation sum from June 1995 to June 1997 ranged from 0 mm to 358 mm while median temperatures varied between 18.4\degree C to 23.5\degree C~(\hyperref[table: climate]{Table S\ref{table: climate}}). From October 2008 until September 2009, monthly precipitation sum ranged from 0 mm to 270.6 mm with median temperatures of 18\degree C to 24\degree C. For analytical purposes, the interactions recorded in the transitional months of April and October were assigned to the dry and rainy seasons respectively. Periods specified for each season are in concordance with patterns reported for other Cerrado areas~\citep{Gottsberger2006a}. 

\subsection{Data analysis}
Due to temporal and spatial differences between the BBG and IBGE datasets, the two datasets were hence analysed separately. 

\subsubsection{Month-to-month turnover}
Bee-flower interaction turnover is calculated using the Whittaker's presence-based dissimilarity measure~\citep{Whittaker1960}: 
\begin{align}
	\beta_{int} & = \frac{a + b + c}{(2a + b + c)/2} - 1 
\label{eq: dissimilarity}
\end{align}
where interaction turnover (i.e. interaction dissimilarity or interaction $\beta$-diversity; $\beta_{int}$) reflects the differences, or dissimilarity, of interactions between two successive monthly networks. $a$ represents the number of interactions present in both networks while b and c are the number of unique interactions in each of the two networks respectively~\citep{Poisot2012}. \\
\\
$\beta_{int}$ can be partitioned into two components; network dissimilarity due to species turnover ($\beta_{st}$) and interaction rewiring between shared species of networks ($\beta_{rw}$):
\begin{align}
	\beta_{int} & = \beta_{st} + \beta_{rw} 
\label{eq: interaction}
\end{align}
In theory, $\beta_{int}$ and $\beta_{st}$, but not $\beta_{rw}$, covary with species turnover, where species turnover reflects the differences between species composition of two networks; $\beta_{S}$. In this study, $\beta_{S}$ can be driven by either plant turnover ($\beta_{Plant}$) or bee turnover ($\beta_{Bee}$). \\
\\
$\beta_{rw}$, $\beta_{S}$, $\beta_{Plant}$ and $\beta_{Bee}$ are calculated using \hyperref[eq: dissimilarity]{Equation \ref{eq: dissimilarity}}, where a refers to the number of items present in both networks and b and c refer to the number of unique items present in each of the two networks~(\autoref{table: dissimilarity}). $\beta_{st}$ is obtained by subtracting $\beta_{rw}$ from $\beta_{int}$~(\hyperref[eq: interaction]{Equation \ref{eq: interaction}}). The dissimilarity measure takes the value of 0 when two networks are identical and the value of 1 when two networks do not share any items in common~\citep{Poisot2012, CaraDonna2017}. \\

%%%%%%%%%%%%%%%%%%%%%%%%%%%%%%%%%%%%%%%%%%%%%%
\begin{table}[h]
\captionof{table}{Measures of network dissimilarity. \\ \doublespacing\footnotesize The contribution of species turnover to interaction turnover is illustrated indirectly by the fraction of interaction turnover due to species turnover alone ($\beta_{st}$). Dissimilarity measures are calculated using the respective items and ~\autoref{eq: dissimilarity}. Modified from \cite{Poisot2012}.}
\label{table: dissimilarity}
\doublespacing
\resizebox{1.07\textwidth}{!}{%
\begin{tabular}{llll}
\hline
\rowcolor[HTML]{FFDA6C} 
Measure & Definition & Items & Reference \\ \hline
$\beta_{int}$ & \begin{tabular}[c]{@{}l@{}}Dissimiliarity of interactions; \\ Interaction turnover\end{tabular} & All interactions & \begin{tabular}[c]{@{}l@{}}\cite{Canard2011}; \\ \cite{CaraDonna2017}  \end{tabular} \\

\rowcolor[HTML]{FFF9F1} 
$\beta_{rw}$ & \begin{tabular}[c]{@{}l@{}}Dissimilarity of interactions between species present in both networks; \\ Interaction rewiring\end{tabular} & \begin{tabular}[c]{@{}l@{}}Interactions of \\ shared species\end{tabular} & \begin{tabular}[c]{@{}l@{}}\cite{Canard2011};\\ \cite{CaraDonna2017} \end{tabular} \\

$\beta_{st}$ & Dissimilarity of interactions due to species turnover & \autoref{eq: dissimilarity} & \cite{Poisot2012}  \\

\rowcolor[HTML]{FFF9F1} 
$\beta_{st}$/$\beta_{int}$ & Contribution of species dissimilarity to dissimilarity of interactions &  & \cite{Poisot2012} \\

$\beta_{S}$ & \begin{tabular}[c]{@{}l@{}}Dissimilarity in the species composition of both networks; \\ Species turnover\end{tabular} & Species identity & e.g. \cite{Koleff2003} \\

\rowcolor[HTML]{FFF9F1} 
$\beta_{Bee}$ & \begin{tabular}[c]{@{}l@{}}Dissimilarity in the bee composition of both networks; \\ Bee turnover\end{tabular} & Bee identity & This study \\

$\beta_{Plant}$ & \begin{tabular}[c]{@{}l@{}}Dissimilarity in the plant composition of both networks; \\ Plant turnover\end{tabular} & Plant identity & This study \\ \hline
\end{tabular}%
}
\end{table}

%%%%%%%%%%%%%%%%%%%%%%%%%%%%%%%%%%%%%%%%%%%%%%

\subsubsection{Correlation}
As turnover measures are dependent variables, the non-parametric Spearman's rank correlation test from the python package SciPy was utilised to investigate the relationships between the different dissimilarity measures as well as the associations of climatic factors and dissimilarity measures~\citep{Dehmer2011}. \\
\\
A Monte Carlo process was then used to generate p-values for correlation tests. p-values of Spearman's test deviate away from actual p-values due to turnovers being dependent variables~(\hyperref[table: turnoverspearman]{Table S\ref{table: turnoverspearman}}). Randomised sets of bees and plants were drawn across the dataset to form 100000 simulated networks for each month. Correlation coefficients between turnover measures for each simulation were thereafter calculated. Number of bees, plants and interactions as well as connectance of each monthly network in simulations were kept constant. p-values were obtained by dividing the total number of simulations with a correlation coefficient higher than the value obtained in either the BBG or IBGE dataset by 100000.

\subsubsection{Climate}
Two climatic models were utilised in this study. The first model uses the differences between precipitations or temperatures of two subsequent months as the explanatory variable of turnovers (hereafter known as the difference model). The alternative hypothesis of the difference model assumes that networks at a particular temperature and precipitation level are static and do not experience changes as long as climatic factors remain constant. When two networks are at the same temperature and precipitation level, interaction turnover equals to zero. Interaction turnover increases as the temperature or precipitation level difference between networks increases.\\
\\
The second model uses the average of precipitations or temperatures of two subsequent months as the explanatory variable of turnovers (hereafter known as the average model). The average model postulates that two networks with identical climatic factors will yield a particular turnover rate. Interaction turnover increases as the temperature or precipitation level of networks increases.\\
\\
To compare the two climatic models, linear regression was used to fit precipitation, temperature and season against $\beta_{Plant}$ and $\beta_{rw}$ values of the BBG dataset. Model fitting was not carried out for the smaller IBGE dataset. The more explanatory model was thereafter used to fit climatic factors against turnovers within seasons to prevent overfitting and to minimise multicollinearity.

\newpage
\section{Results} % 1800 words
\subsection{Community composition}
111 species of bees and 93 species of plants were recorded over the 12-month study period at IBGE. In total, 968 bee-flower interactions, which comprised of 434 unique interactions, were observed. The bee community composition in IBGE was similar to those previously observed in other Cerrado areas~\citep{Silveira1995, Pinheiro-Machado2002} with Apidae being the richest group (77 spp.) followed by Halictidae (19 spp.), Megachilidae (13 spp.), Andrenidae (1 sp.), and Colletidae (1sp.). Plant species recorded at this site consisted of mainly herbs and shrubs, and belonged to 24 families, the most species rich group being Fabaceae (18 spp.). \\
\\
Between June 1995 and June 1997, 1050 unique interactions and 1616 visitation events between 203 bee species and 182 plant were recorded in the \textit{cerrado sensu strictu} area of BBG. Although the BBG area contained a more species rich pollinator community, bee families were present in comparable proportions as those observed in IBGE: Apidae (115 spp.), Halictidae (38 spp.), Megachilidae (27 spp.), Colletidae (3 spp.), and Andrenidae (1 sp.). Plants recorded in BBG represented 41 families, consisting of mostly shrubs, some trees, and a few herbs. Similar to IBGE, Fabaceae was the most species rich group in this area (31 spp.), followered by Asteraceae (20 spp.) and Malpighiaceae (17 spp.). 
% generalised?

\subsection{Month-to-month turnover}

Interaction turnover, $\beta_{int}$, is consistently high, ranging from 0.747 to 1~(\hyperref[table: turnovers]{Table S\ref{table: turnovers}}). As expected, $\beta_{int}$ is positively correlated with $\beta_{S}$~(BBG: $r_{s}$=0.698, $p$=0.0308; IBGE: $r_{s}$=0.809, $p$=0.0123) as an increase in species turnover, $\beta_{S}$, will drive an increase in $\beta_{int}$. At both sites, $\beta_{int}$ is significantly and positively correlated with $\beta_{Plant}$~(BBG: $r_{s}$=0.822, $p$=0.00008; IBGE: $r_{s}$=0.773, $p$=0.0118), suggesting that $\beta_{Plant}$ drives $\beta_{int}$.\\
\\
Moreover, there is a relatively weak and non-significant correlation between interaction rewiring, $\beta_{rw}$, and $\beta_{S}$~(BBG: $r_{s}$=-0.444, $p$=0.176; IBGE: $r_{s}$=0.629, $p$=0.359), indicating that factors driving $\beta_{rw}$ are different from those that drive $\beta_{S}$.\\
\\
Although there is a high correlation value between bee turnover, $\beta_{Bee}$, and plant turnover, $\beta_{Plant}$, both trends occur by chance and are statistically non-significant. At the BBG site, neither $\beta_{Bee}$ nor $\beta_{Plant}$ drives $\beta_{S}$~(\hyperref[table: turnovers]{Table S\ref{table: turnovers}}). However, $\beta_{S}$ has a strong and significant positive correlation with $\beta_{Plant}$ at the IBGE site~($r_{s}$=0.964, $p$=0.0021), suggesting that plants are the main driver of species turnover, $\beta_{S}$, at this site.\\
\\
Surprisingly, $\beta_{st}$ does not associate with $\beta_{S}$ at both sites~(\hyperref[table: turnovers]{Table S\ref{table: turnovers}}). $\beta_{st}$ indirectly reflects the contribution of $\beta_{S}$ to $\beta_{int}$ and will theoretically increase as $\beta_{S}$ increases. However, due to insufficient sampling and climate conditions, interactions were unequally sampled across time, resulting in inflated $\beta_{rw}$ values. As $\beta_{st}$ is obtained by subtracting $\beta_{rw}$ from $\beta_{int}$, this results in $\beta_{st}$ values being underestimated and the lack of relationship between $\beta_{st}$ and $\beta_{S}$. Hence, $\beta_{rw}$ and $\beta_{st}$ will hereafter not be used for analysis. Nonetheless, $\beta_{int}$, $\beta_{S}$, $\beta_{Plant}$ and $\beta_{Bee}$ accumulate less error than $\beta_{rw}$ and are more robust to sampling efforts, allowing these measures to be appropriate for further analysis~\citep{Poisot2012}.\\


\subsection{Climatic factors influence turnover}


\newpage
\section{Discussion} % 1800 words
Discuss results, why climatic factor would explain turnover \\
Limitations: Correlation does not relate to causation, but diff to conduct such experiments,  \\
Future Research: more data, artic temperature comparisons \\

% acknowledgement page
\newpage 
\vspace*{\fill}

{\huge\bfseries Acknowledgements} \label{sec: acknowledgements} \\
\\
\\
\large{I would like to thank Dr. Samraat Pawar for his supervision and guidance throughout this project, as well as  all of my friends who have helped me in one way or another with Python programming. Finally, I would like to thank M. C. Boaventura and S. C. Cappellari for making their datasets available. Field data collection at IBGE was made possible with the support of P. H. Pinheiro, the staff of Reserva Ecol\'{o}gica do IBGE and the graduate program in Ecology at the University of Brasilia. The following experts contributed towards the identification of plants: M. A. da Silva, M. C. Mamede, C. Proen\c{c}a, A. L. Prado, S. L. Silva, L. P. Queiroz, A. Krapovikas, L. F. Oliveira, T. B. Cavalcante, K. Calago, A. E. Ramos, C. Munhoz, F. Silva, M. G. N\'{o}brega, R. C. Martins, and R. C. Oliveira, and the following experts for bee identifications: A. J. C. Aguiar, A. Raw, M. C. Boaventura, G. A. R. Melo, F. Vivallo, and D. Urban.}
\vfill

\newpage
\bibliography{buzz}
\bibliographystyle{cell}

\newpage
\section{Supplementary Figures}
%%%%%%%%%% Merge with supplemental materials %%%%%%%%%%
%%%%%%%%%% Prefix a "S" to all equations, figures, tables and reset the counter %%%%%%%%%%
\setcounter{equation}{0}
\setcounter{figure}{0}
\setcounter{table}{0}
\makeatletter
\renewcommand{\theequation}{S\arabic{equation}}
\renewcommand{\thefigure}{S\arabic{figure}}
\renewcommand{\bibnumfmt}[1]{[S#1]}
\renewcommand{\citenumfont}[1]{S#1}
%%%%%%%%%% Prefix a "S" to all equations, figures, tables and reset the counter %%%%%%%%%%

%%%%%%%%%%%%%%%%%%%%%%%%%%%%%%%%%%%%%%%%%%%%%%
\begin{table}[h]
\centering
\setstretch{1.3}
\captionof{table}{Month-to-month turnover values for all dissimilarity measures at both the Bras\'ilia's Botanical Garden Protected Area (BBG) and Reserva Ecol\'ogica do IBGE (IBGE) sites.}
\label{table: turnovers}
\resizebox{\textwidth}{!}{%
\begin{tabular}{|c|c|r|r|r|r|r|r|r|r|c|}
\hline
\rowcolor[HTML]{EFEFEF} 
Year & Months  & \multicolumn{1}{c|}{$\beta_{int}$}   & \multicolumn{1}{c|}{$\beta_{rw}$}    & \multicolumn{1}{c|}{$\beta_{st}$}    &\multicolumn{1}{c|}{$\beta_{rw}$/$\beta_{int}$}  & \multicolumn{1}{c|}{$\beta_{st}$/$\beta_{int}$}  &  \multicolumn{1}{c|}{$\beta_{S}$}     &  \multicolumn{1}{c|}{$\beta_{Plant}$} &  \multicolumn{1}{c|}{$\beta_{Bee}$}    & Site \\ \hline
1995 & Jun-Jul & 1     & 0     & 1     & 0      & 1      & 0.895 & 0.8   & 1     & BBG  \\ \hline
1995 & Jul-Aug & 0.967 & 0.667 & 0.301 & 0.689  & 0.311  & 0.714 & 0.742 & 0.68  & BBG  \\ \hline
1995 & Aug-Sep & 0.931 & 0.7   & 0.231 & 0.752  & 0.248  & 0.59  & 0.542 & 0.657 & BBG  \\ \hline
1995 & Sep-Oct & 0.921 & 0.538 & 0.383 & 0.585  & 0.415  & 0.658 & 0.659 & 0.657 & BBG  \\ \hline
1995 & Oct-Nov & 0.959 & 0.333 & 0.626 & 0.348  & 0.652  & 0.787 & 0.76  & 0.818 & BBG  \\ \hline
1995 & Nov-Dec & 0.857 & 0     & 0.857 & 0      & 1      & 0.81  & 0.857 & 0.714 & BBG  \\ \hline
1995 & Dec-Jan & 0.846 & 0     & 0.846 & 0      & 1      & 0.789 & 0.846 & 0.667 & BBG  \\ \hline
1996 & Jan-Feb & 1     & 0     & 1     & 0      & 1      & 0.951 & 0.923 & 1     & BBG  \\ \hline
1996 & Feb-Mar & 0.93  & 0.5   & 0.43  & 0.538  & 0.462  & 0.781 & 0.76  & 0.826 & BBG  \\ \hline
1996 & Mar-Apr & 0.857 & 0.517 & 0.34  & 0.603  & 0.397  & 0.567 & 0.577 & 0.538 & BBG  \\ \hline
1996 & Apr-May & 0.859 & 0.667 & 0.192 & 0.776  & 0.224  & 0.527 & 0.495 & 0.6   & BBG  \\ \hline
1996 & May-Jun & 0.747 & 0.375 & 0.372 & 0.502  & 0.498  & 0.503 & 0.48  & 0.547 & BBG  \\ \hline
1996 & Jun-Jul & 0.781 & 0.556 & 0.226 & 0.711  & 0.289  & 0.476 & 0.468 & 0.49  & BBG  \\ \hline
1996 & Jul-Aug & 0.821 & 0.509 & 0.312 & 0.62   & 0.38   & 0.538 & 0.506 & 0.6   & BBG  \\ \hline
1996 & Aug-Sep & 0.973 & 0.75  & 0.223 & 0.771  & 0.229  & 0.635 & 0.529 & 0.854 & BBG  \\ \hline
1996 & Sep-Oct & 0.965 & 0.733 & 0.232 & 0.76   & 0.24   & 0.607 & 0.512 & 0.867 & BBG  \\ \hline
1996 & Oct-Nov & 0.874 & 0.486 & 0.388 & 0.556  & 0.444  & 0.6   & 0.564 & 0.724 & BBG  \\ \hline
1996 & Nov-Dec & 0.835 & 0.483 & 0.352 & 0.578  & 0.422  & 0.541 & 0.5   & 0.676 & BBG  \\ \hline
1996 & Dec-Jan & 0.987 & 0.867 & 0.12  & 0.878  & 0.122  & 0.635 & 0.567 & 0.765 & BBG  \\ \hline
1997 & Jan-Feb & 0.881 & 0.6   & 0.281 & 0.681  & 0.319  & 0.563 & 0.531 & 0.607 & BBG  \\ \hline
1997 & Feb-Mar & 0.867 & 0.621 & 0.246 & 0.716  & 0.284  & 0.622 & 0.628 & 0.614 & BBG  \\ \hline
1997 & Mar-Apr & 0.817 & 0.621 & 0.196 & 0.76   & 0.24   & 0.503 & 0.456 & 0.593 & BBG  \\ \hline
1997 & Apr-May & 0.818 & 0.52  & 0.298 & 0.636  & 0.364  & 0.544 & 0.515 & 0.6   & BBG  \\ \hline
1997 & May-Jun & 0.897 & 0.333 & 0.563 & 0.372  & 0.628  & 0.754 & 0.692 & 0.846 & BBG  \\ \hline
2008 & Oct-Nov & 1     & 0     & 1     & 0      & 1      & 0.909 & 0.86  & 1     & IBGE \\ \hline
2008 & Nov-Dec & 0.862 & 0.5   & 0.362 & 0.58   & 0.42   & 0.617 & 0.581 & 0.688 & IBGE \\ \hline
2008 & Dec-Jan & 0.894 & 0.611 & 0.283 & 0.684  & 0.316  & 0.597 & 0.543 & 0.673 & IBGE \\ \hline
2009 & Jan-Feb & 0.836 & 0.474 & 0.362 & 0.567  & 0.433  & 0.582 & 0.576 & 0.591 & IBGE \\ \hline
2009 & Feb-Mar & 0.934 & 0.613 & 0.321 & 0.656  & 0.344  & 0.659 & 0.636 & 0.692 & IBGE \\ \hline
2009 & Mar-Apr & 0.966 & 0.829 & 0.138 & 0.858  & 0.142  & 0.701 & 0.703 & 0.698 & IBGE \\ \hline
2009 & Apr-May & 0.895 & 0.333 & 0.561 & 0.373  & 0.627  & 0.66  & 0.613 & 0.727 & IBGE \\ \hline
2009 & May-Jun & 0.902 & 0.5   & 0.402 & 0.555  & 0.445  & 0.667 & 0.6   & 0.75  & IBGE \\ \hline
2009 & Jun-Jul & 0.753 & 0.333 & 0.42  & 0.442  & 0.558  & 0.564 & 0.707 & 0.405 & IBGE \\ \hline
2009 & Jul-Aug & 0.761 & 0.429 & 0.333 & 0.563  & 0.437  & 0.514 & 0.529 & 0.5   & IBGE \\ \hline
2009 & Aug-Sep & 0.87  & 0.143 & 0.727 & 0.164  & 0.836  & 0.808 & 0.76  & 0.852 & IBGE \\ \hline
\end{tabular}%
}
\end{table}

%%%%%%%%%%%%%%%%%%%%%%%%%%%%%%%%%%%%%%%%%%%%%%
%%%%%%%%%%%%%%%%%%%%%%%%%%%%%%%%%%%%%%%%%%%%%%

\begin{table}[]
\centering
\setstretch{1.3}
\captionof{table}{Climate information obtained from IBGE's weather station. \\
\footnotesize Daily average temperature was calculated using the minimum and maximum temperature of each day. The median value of daily temperatures was then obtained for each month and used for data analysis. Precipitation values were acquired by adding together total amount of rainfall that occurred throughout the month.}
\label{table: climate}
\resizebox{0.6\textwidth}{!}{%
\begin{tabular}{|r|c|r|r|}
\hline
\rowcolor[HTML]{EFEFEF} 
Year & Month & Precipitation / mm & Temperature / \degree C \\ \hline
1995 & Jun & 0 & 18.7 \\ \hline
1995 & Jul & 0 & 19.4 \\ \hline
1995 & Aug & 0 & 20.9 \\ \hline
1995 & Sep & 5 & 22.4 \\ \hline
1995 & Oct & 86.5 & 23.5 \\ \hline
1995 & Nov & 292.1 & 21.6 \\ \hline
1995 & Dec & 328.3 & 21.9 \\ \hline
1996 & Jan & 66 & 22.3 \\ \hline
1996 & Feb & 190 & 22.3 \\ \hline
1996 & Mar & 304.7 & 22.4 \\ \hline
1996 & Apr & 39.7 & 21.2 \\ \hline
1996 & May & 26.5 & 20.4 \\ \hline
1996 & Jun & 0 & 18.4 \\ \hline
1996 & Jul & 0 & 18.7 \\ \hline
1996 & Aug & 48 & 20.4 \\ \hline
1996 & Sep & 37.6 & 22.4 \\ \hline
1996 & Oct & 116.3 & 23 \\ \hline
1996 & Nov & 261.2 & 22.1 \\ \hline
1996 & Dec & 293.6 & 22.7 \\ \hline
1997 & Jan & 358 & 21.7 \\ \hline
1997 & Feb & 104.3 & 22.1 \\ \hline
1997 & Mar & 304.2 & 21.6 \\ \hline
1997 & Apr & 148.1 & 21.2 \\ \hline
1997 & May & 114.2 & 19 \\ \hline
1997 & Jun & 21.4 & 18.8 \\ \hline
2008 & Oct & 29.8 & 24 \\ \hline
2008 & Nov & 196.9 & 22.2 \\ \hline
2008 & Dec & 270.6 & 22.1 \\ \hline
2009 & Jan & 179.8 & 22.3 \\ \hline
2009 & Feb & 153.3 & 22.1 \\ \hline
2009 & Mar & 171.7 & 22.2 \\ \hline
2009 & Apr & 225.2 & 21.3 \\ \hline
2009 & May & 43.3 & 19.4 \\ \hline
2009 & Jun & 26.2 & 18 \\ \hline
2009 & Jul & 0 & 19 \\ \hline
2009 & Aug & 31.9 & 19.8 \\ \hline
2009 & Sep & 36.2 & 22.9 \\ \hline
\end{tabular}%
}
\end{table}

%%%%%%%%%%%%%%%%%%%%%%%%%%%%%%%%%%%%%%%%%%%%%%
%%%%%%%%%%%%%%%%%%%%%%%%%%%%%%%%%%%%%%%%%%%%%%

\begin{table}[h]
\centering
\setstretch{1.3}
\captionof{table}{Relationships between turnover measures. \\
(\footnotesize $r_{s}$: Spearman's correlation coefficient; $p_{s}$: p-value of Spearman's test; p-value: value generated using 100000 randomised networks for each month)}
\label{table: turnoverspearman}
\resizebox{\textwidth}{!}{%
\begin{tabular}{|l|l|r|r|r|l|l|l|r|r|r|}
\cline{1-5} \cline{7-11}
\multicolumn{5}{|c|}{\cellcolor[HTML]{EFEFEF}BBG site, Cerrado (1995-1997)} &  & \multicolumn{5}{c|}{\cellcolor[HTML]{EFEFEF}IBGE site, Cerrado (2008-2009)} \\ \cline{1-5} \cline{7-11} 
\multicolumn{2}{|c|}{\cellcolor[HTML]{EFEFEF}Measures} & \multicolumn{1}{c|}{\cellcolor[HTML]{EFEFEF}$r_{s}$} & \multicolumn{1}{c|}{\cellcolor[HTML]{EFEFEF}$p_{s}$} & \multicolumn{1}{c|}{\cellcolor[HTML]{EFEFEF}p-value} &  & \multicolumn{2}{c|}{\cellcolor[HTML]{EFEFEF}Measures} & \multicolumn{1}{c|}{\cellcolor[HTML]{EFEFEF}$r_{s}$} & \multicolumn{1}{c|}{\cellcolor[HTML]{EFEFEF}$p_{s}$} & \multicolumn{1}{c|}{\cellcolor[HTML]{EFEFEF}p-value} \\ \cline{1-5} \cline{7-11} 
$\beta_{int}$ & $\beta_{st}$ & 0.140 & 0.514 & 0.145 &  & $\beta_{int}$ & $\beta_{st}$ & 0.140 & 0.514 & 0.037 \\ \cline{1-5} \cline{7-11} 
$\beta_{int}$ & $\beta_{rw}$ & 0.156 & 0.468 & 0.509 &  & $\beta_{int}$ & $\beta_{rw}$ & 0.156 & 0.468 & 0.844 \\ \cline{1-5} \cline{7-11} 
$\beta_{int}$ & $\beta_{S}$ & 0.698 & 0 & 0.031 &  & $\beta_{int}$ & $\beta_{S}$ & 0.698 & 0.000 & 0.012 \\ \cline{1-5} \cline{7-11} 
$\beta_{int}$ & $\beta_{Bee}$ & 0.568 & 0.004 & 0.150 &  & $\beta_{int}$ & $\beta_{Bee}$ & 0.568 & 0.004 & 0.364 \\ \cline{1-5} \cline{7-11} 
$\beta_{int}$ & $\beta_{Plant}$ & 0.822 & 0 & 0.0001 &  & $\beta_{int}$ & $\beta_{Plant}$ & 0.822 & 0 & 0.012 \\ \cline{1-5} \cline{7-11} 
$\beta_{st}$ & $\beta_{S}$ & 0.678 & 0.0003 & 0.177 &  & $\beta_{st}$ & $\beta_{S}$ & 0.678 & 0.0003 & 0.379 \\ \cline{1-5} \cline{7-11} 
$\beta_{st}$ & $\beta_{Bee}$ & 0.732 & 0.0001 & 0.056 &  & $\beta_{st}$ & $\beta_{Bee}$ & 0.732 & 0.0001 & 0.158 \\ \cline{1-5} \cline{7-11} 
$\beta_{st}$ & $\beta_{Plant}$ & 0.425 & 0.038 & 0.640 &  & $\beta_{st}$ & $\beta_{Plant}$ & 0.425 & 0.038 & 0.191 \\ \cline{1-5} \cline{7-11} 
$\beta_{S}$ & $\beta_{Bee}$ & 0.955 & 0 & 0.097 &  & $\beta_{S}$ & $\beta_{Bee}$ & 0.955 & 0 & 0.873 \\ \cline{1-5} \cline{7-11} 
$\beta_{S}$ & $\beta_{Plant}$ & 0.791 & 0 & 0.391 &  & $\beta_{S}$ & $\beta_{Plant}$ & 0.791 & 0 & 0.002 \\ \cline{1-5} \cline{7-11} 
$\beta_{Plant}$ & $\beta_{Bee}$ & 0.610 & 0.002 & 0.242 &  & $\beta_{Plant}$ & $\beta_{Bee}$ & 0.610 & 0.002 & 0.116 \\ \cline{1-5} \cline{7-11} 
\end{tabular}%
}
\end{table}

%%%%%%%%%%%%%%%%%%%%%%%%%%%%%%%%%%%%%%%%%%%%%%
%%%%%%%%%%%%%%%%%%%%%%%%%%%%%%%%%%%%%%%%%%%%%%

%\begin{table}[]
%\centering
%\setstretch{1.3}
%\captionof{table}{Relationships between turnover measures. \\
%(\footnotesize $r_{s}$: Spearman's correlation coefficient; $p_{s}$: p-value of Spearman's test; p-value: value generated using 100000 randomised networks for each month)}
%\label{table: turnoverspearman}
%%\resizebox{\textwidth}{!}{%
%\begin{tabular}{|l|l|r|r|r|}
%\hline
%\rowcolor[HTML]{EFEFEF} 
%\multicolumn{5}{|c|}{\cellcolor[HTML]{EFEFEF}BBG site, Cerrado (1995-1997)} \\ \hline
%\rowcolor[HTML]{EFEFEF} 
%\multicolumn{1}{|c|}{\cellcolor[HTML]{EFEFEF}Measure 1} & \multicolumn{1}{c|}{\cellcolor[HTML]{EFEFEF}Measure 2} & \multicolumn{1}{c|}{\cellcolor[HTML]{EFEFEF}$r_{s}$} & \multicolumn{1}{c|}{\cellcolor[HTML]{EFEFEF}$p_{s}$} & \multicolumn{1}{c|}{\cellcolor[HTML]{EFEFEF}p-value} \\ \hline
%$\beta_{int}$ & $\beta_{st}$ & 0.140 & 0.514 & 0.145 \\ \hline
%$\beta_{int}$ & $\beta_{rw}$ & 0.156 & 0.468 & 0.509 \\ \hline
%$\beta_{int}$ & $\beta_{S}$ & 0.698 & 0 & 0.031 \\ \hline
%$\beta_{int}$ & $\beta_{Bee}$ & 0.568 & 0.004 & 0.150 \\ \hline
%$\beta_{int}$ & $\beta_{Plant}$ & 0.822 & 0 & 0.0001 \\ \hline
%$\beta_{st}$ & $\beta_{S}$ & 0.678 & 0.0003 & 0.177 \\ \hline
%$\beta_{st}$ & $\beta_{Bee}$ & 0.732 & 0.0001 & 0.056 \\ \hline
%$\beta_{st}$ & $\beta_{Plant}$ & 0.425 & 0.038 & 0.640 \\ \hline
%$\beta_{S}$ & $\beta_{Bee}$ & 0.955 & 0 & 0.097 \\ \hline
%$\beta_{S}$ & $\beta_{Plant}$ & 0.791 & 0 & 0.391 \\ \hline
%$\beta_{Plant}$ & $\beta_{Bee}$ & 0.610 & 0.002 & 0.242 \\ \hline
%\rowcolor[HTML]{EFEFEF} 
%\multicolumn{5}{|c|}{\cellcolor[HTML]{EFEFEF}IBGE site, Cerrado (2008-2009)} \\ \hline
%\rowcolor[HTML]{EFEFEF} 
%\multicolumn{1}{|c|}{\cellcolor[HTML]{EFEFEF}Measure 1} & \multicolumn{1}{c|}{\cellcolor[HTML]{EFEFEF}Measure 2} & \multicolumn{1}{c|}{\cellcolor[HTML]{EFEFEF}$r_{s}$} & \multicolumn{1}{c|}{\cellcolor[HTML]{EFEFEF}$p_{s}$} & \multicolumn{1}{c|}{\cellcolor[HTML]{EFEFEF}p-value} \\ \hline
%$\beta_{int}$ & $\beta_{st}$ & 0.140 & 0.514 & 0.037 \\ \hline
%$\beta_{int}$ & $\beta_{rw}$ & 0.156 & 0.468 & 0.844 \\ \hline
%$\beta_{int}$ & $\beta_{S}$ & 0.698 & 0.000 & 0.012 \\ \hline
%$\beta_{int}$ & $\beta_{Bee}$ & 0.568 & 0.004 & 0.364 \\ \hline
%$\beta_{int}$ & $\beta_{Plant}$ & 0.822 & 0 & 0.012 \\ \hline
%$\beta_{st}$ & $\beta_{S}$ & 0.678 & 0.0003 & 0.379 \\ \hline
%$\beta_{st}$ & $\beta_{Bee}$ & 0.732 & 0.0001 & 0.158 \\ \hline
%$\beta_{st}$ & $\beta_{Plant}$ & 0.425 & 0.038 & 0.191 \\ \hline
%$\beta_{S}$ & $\beta_{Bee}$ & 0.955 & 0 & 0.873 \\ \hline
%$\beta_{S}$ & $\beta_{Plant}$ & 0.791 & 0 & 0.002 \\ \hline
%$\beta_{Plant}$ & $\beta_{Bee}$ & 0.610 & 0.002 & 0.116 \\ \hline
%\end{tabular}%
%%}
%\end{table}

%%%%%%%%%%%%%%%%%%%%%%%%%%%%%%%%%%%%%%%%%%%%%%
%%%%%%%%%%%%%%%%%%%%%%%%%%%%%%%%%%%%%%%%%%%%%%

\end{document}
