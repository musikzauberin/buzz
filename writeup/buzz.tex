\documentclass[11pt]{article}
\usepackage{geometry}
\geometry{
 a4paper,
 total={210mm,297mm},
 left=25mm, 
 right=25mm,
 top=25mm,
 bottom=25mm,
 }
 
\usepackage[english]{babel}
\usepackage[utf8x]{inputenc}
\usepackage{amsmath}
\usepackage{graphicx}
\usepackage{multicol}
\usepackage{ragged2e}
\usepackage{tocloft}
\usepackage{gensymb}
\usepackage{siunitx}
\usepackage[hypcap]{caption}
\usepackage{capt-of}
\usepackage{setspace}
\usepackage[parfill]{parskip}
\usepackage{amssymb}
\usepackage{textcomp}
\usepackage{float}
\floatstyle{plain}
\usepackage{color}
\usepackage{epstopdf}
\usepackage{natbib}
\usepackage{hyperref}
\hypersetup{
    colorlinks=true,
    linkcolor=blue,
    filecolor=magenta,      
    urlcolor=cyan,
    citecolor=blue
}
\urlstyle{same}
\usepackage{cleveref}
\usepackage{lastpage}
\usepackage{fancyhdr}

%abstract setup
\renewenvironment{abstract}
 {\hspace{.8cm}
  {\bfseries\huge\abstractname}
  \list{}{
    \setlength{\leftmargin}{.95cm}%
    \setlength{\rightmargin}{\leftmargin}%
  }%
  \item\relax}
 {\endlist}

%random things needed
%\renewcommand{\cftsecleader}{\cftdotfill{\cftdotsep}} %let the content table have dots to number
\linespread{1.6}  %one and half is 1.3, doublespacing is 1.6
\frenchspacing


\begin{document}

\begin{titlepage}
	\centering
	\topskip0pt
	\vspace*{\fill}
	{\huge\bfseries Temporal turnover of plant-pollinator interaction networks \par}
	\vspace{2cm}
	{\Large \textsc{Lim} Jia Le  {    }  CID: 00865029}
	\\ 	\vspace{0.5cm}
	{Department of Biology, Imperial College London, \\Silwood Campus, London, U.K.} \\ \vspace{0.5cm}
	{Submitted in part fulfilment of the requirements for the \\ Bachelor of Science degree in Biology with German for Science \\ of Imperial College London.} \\
	\vspace*{\fill}
	{\large Supervised by\par
	Dr.~Samraat \textsc{Pawar}}
	\vfill
% Bottom of the page
	{\large Last updated: \today\par}
\end{titlepage}

%% acknowledgement page
%\newpage 
%\pagenumbering{gobble}
%\vspace*{\fill}
%{\huge\bfseries Acknowledgements} \\
%\\
%\\
%\large{I would like to thank }
%\vfill


\newpage
\pagenumbering{gobble}
\vspace*{\fill}
\begin{abstract}  
\doublespacing
abstracting abstracting abstracting

\end{abstract}
\vfill

% contents page
\newpage
\pagenumbering{gobble}
\vspace*{\fill}
\tableofcontents 
\vspace*{\fill} 
\thispagestyle{empty}

\doublespacing

% abbreviations page?
\newpage 
\vspace*{\fill}
{\huge\bfseries Abbreviations} \\
\\
\\
\large{maybe a table}
\vfill

% Page number
\newpage
\pagestyle{fancy}
\fancyhf{}
\renewcommand{\headrulewidth}{0pt}
\rfoot{Page \thepage \hspace{1pt} of \pageref{LastPage}}
\pagenumbering{arabic}

%%% Start writing
\section{Introduction} 




\newpage
\section{Materials and Methods} 
\subsection{Study area and climate information}
The Cerrado spans across most of Central Brazil while extending marginally into Bolivia and Paraguay. It comprises of vegetation ranging from open grasslands to scrublands with a sparse distribution of trees, and smaller regions of gallery and close canopy forests. These patches exist side-by-side, resulting in a highly heterogeneous ecosystem~\citep{Gottsberger2006}. \\
\\
Plant-pollinator interactions were surveyed in the Reserva Ecol\'ogica do IBGE (hereafter `IBGE') and in the Protected Area of the Jardim Bot\^anico de Bras\'ilia (Bras\'ilia's Botanical Garden Protected Area; hereafter `BBG'). The two study sites are located on the Brazilian plateau (1,100m a.s.l.), within the federally protected conservation site ``APA-Gama-Cabe\c ca-de-Viado,'', located approximately 30km south of Bras\'ilia (15\degree56'S, 47\degree53'W). This region is characterized by a well-defined wet summer season that lasts from November until March followered by a dry winter period that extends from May until September.  \\
\\
Data from the IBGE's weather station was used to obtain monthly median temperature and precipitation sum for the past 30 years (1980-2010). Monthly precipitation sum from June 1995 to June 1997 ranged from ? to ? while median temperatures varied between ? to ?. From October 2008 until September 2009, monthly precipitation sum ranged from ? to ? with median temperatures of ? to ?.\\

\subsection{Sampling methods and species identification}
Bees are the predominant pollinators in the Cerrado ($\sim$70\%) followed by moths ($\sim$12\%), hummingbirds ($\sim$3\%), bats, ($\sim$2\%) and beetles ($\sim$2\%)~\citep{Oliveira2002, Gottsberger2006a, Cappellari2011}. Hence, this study focused only on bee-flower interactions. A plant or pollinator species was included in the surveys only if the flowering plant received visits or if the bee was seen foraging on flowers. For every interaction observed, the plant was tagged with a unique identification number, photographed and vouchered. Plant vouchers were identified by using comparative herbarium material, a checklist of local angiosperms and local botanical expertise.  In both sites, bees were collected with an entomological net and killed either in individual vials with paper pellets moistened with ethyl acetate or frozen after each observation. Insect vouchers were thereafter mounted, preserved, and identified to species level by comparison with reference collections, taxonomic literature, local records~\citep{Moure1962, Silveira2002, Moure2007} and by local entomological experts. 




\subsection{Data Analyses}

\subsubsection{Turnover Analyses}
\subsubsection{Climate Analyses}
\subsubsection{Correlation Analyses}

\newpage
\section{Results} 
\subsection{Community composition}
93 species of plants and 111 species of bees were recorded over the 12-month study period at IBGE. In total, 968 bee-flower interactions, which comprised of 434 unique interactions, were observed. The bee community composition in IBGE was similar to those previously observed in other Cerrado areas~\citep{Silveira1995, Pinheiro-Machado2002} with Apidae being the richest group (77 spp.) followed by Halictidae (19 spp.), Megachilidae (13 spp.), Andrenidae (1 sp.), and Colletidae (1sp.). Plant species recorded at this site consisted of mainly herbs and shrubs, and belonged to 24 families, the most species rich group being Fabaceae (18 spp.). \\
\\
Between June 1995 and June 1997, ? unique interactions and ? visitation events between ? plant and ? bee species were recorded in the \textit{cerrado sensu strictu} area of BBG. Although the BBG area contained a more species rich pollinator community, bee families were present in comparable proportions as those observed in IBGE: Apidae (115 spp.), Halictidae (38 spp.), Megachilidae (27 spp.), Colletidae (3 spp.), and Andrenidae (1 sp.). Plants recorded in BBG represented 41 families, consisting of mostly shrubs, some trees, and a few herbs. Similar to IBGE, Fabaceae was the most species rich group in this area (31 spp.), followered by Asteraceae (20 spp.) and Malpighiaceae (17 spp.). 

\subsection{High turnover}
\subsection{Climatic factors influence turnover}

\newpage
\section{Discussion} 


\newpage
\bibliography{CRISPR2}
\bibliographystyle{cell}

\section{Supplementary Figures}

%Appendices/Supplementary Information (optional)


\end{document}
