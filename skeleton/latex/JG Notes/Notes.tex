\documentclass[12pt]{article}
\usepackage[a4paper, left = 2cm, right = 2cm, top = 3cm, bottom = 2cm, footskip = 0.7cm]{geometry}
\usepackage{parskip}
\usepackage[english]{babel}
\usepackage{mathtools}
\usepackage{amssymb}
\usepackage{mathdots}
\usepackage{wrapfig}
\usepackage{changepage}
\usepackage{enumitem}
\usepackage[usenames, dvipsnames]{xcolor}
\usepackage{hyperref}
\usepackage{sectsty}
\usepackage[normalem]{ulem}
\usepackage{lastpage}
\usepackage{fancyhdr}
%\usepackage{showframe}
\setlength{\arraycolsep}{2.4pt}
\def \arraystretch{0.7}
\setlist{nosep, leftmargin = *}
\setlength{\parsep}{0pt}
\setlength{\intextsep}{0pt}
\linespread{1.5}
\everymath{\displaystyle}
\allowdisplaybreaks

\newcommand{\noparskip}{\vspace{-\parskip}}
\newenvironment{points}
	{\begin{enumerate}[label = (\arabic*)]}
	{\end{enumerate}}
\newenvironment{dent}
	{\begin{adjustwidth}{15pt}{}\noparskip}
	{\end{adjustwidth}}
\newenvironment{result}[1]
	{\textcolor{Red}{\textbf{#1:}}\begin{dent}}
	{\end{dent}}
\newenvironment{proof}[1]
	{\textcolor{Orange}{\textbf{Proof of #1:}}\begin{dent}}
	{\end{dent}}
\newenvironment{definition}
	{\textcolor{Blue}{\textbf{Definition:}}\begin{dent}}
	{\end{dent}}
\newenvironment{notation}
	{\textcolor{Plum}{\textbf{Notation:}}\begin{dent}}
	{\end{dent}}
\newenvironment{example}
	{\textcolor{Green}{\textbf{Example:}}\begin{dent}}
	{\end{dent}}
\newenvironment{remark}
	{\textcolor{Brown}{\textbf{Remark:}}\begin{dent}}
	{\end{dent}}
\newcommand{\pic}[2][1.0]{
	$\vcenter{\hbox{\includegraphics[scale = #1]{#2}\ }}$}
\newcommand{\wrappic}[2][1.0]{
	\begin{wrapfigure}{r}{0pt}
	\centering
	\raisebox{-3\baselineskip}{\pic[#1]{#2}}
	\end{wrapfigure}}

\renewcommand{\implies}{\Rightarrow}
\renewcommand{\iff}{\Leftrightarrow}
\newcommand{\contradiction}{\Rightarrow \Leftarrow}
\newcommand{\set}[1]{\left\{ #1 \right\}}
\newcommand{\sizeof}[1]{\left| #1 \right|}
\newcommand{\Z}{\mathbb{Z}}
\newcommand{\N}{\mathbb{N}}
\newcommand{\Q}{\mathbb{Q}}
\newcommand{\R}{\mathbb{R}}
\newcommand{\Ztwox}{\Z_2[x]}
\newcommand{\Ztwoxover}[1]{\frac{\Ztwox}{#1}}
\newcommand{\gen}[1]{\left( #1 \right)}
\newcommand{\Ztwoxgen}[1]{\frac{\Ztwox}{\gen{#1}}}
\newcommand{\xbar}{\bar{x}}
\newcommand{\cB}{\mathcal{B}}
\newcommand{\field}[1]{\mathbb{F}_{#1}}
\newcommand{\cycgen}[1]{\langle #1 \rangle}
\newcommand{\rank}{\operatorname{rank}}
\newcommand{\wt}[1]{\operatorname{wt}(#1)}
\newcommand{\Ham}[1]{\operatorname{Ham}(#1)}
\newcommand{\ord}[1]{\operatorname{ord}(#1)}
\newcommand{\lcm}[1]{\operatorname{lcm}\set{#1}}
\newcommand{\floor}[1]{\left\lfloor #1 \right\rfloor}
\newcommand{\diam}[1]{\operatorname{diam}(#1)}
\renewcommand{\ker}{\operatorname{Ker}}
\newcommand{\img}{\operatorname{Im}}
\newcommand{\trace}[1]{\operatorname{Trace}(#1)}
\renewcommand{\span}[1]{\operatorname{Sp}(#1)}
\renewcommand{\vec}[1]{\mathbf{#1}}
\newcommand{\cL}{\mathcal{L}}
\newcommand{\aut}[1]{\operatorname{Aut}(#1)}

\setlength{\headheight}{15pt}
\setlength{\headsep}{12pt}
\fancyhead[L]{}
\fancyhead[R]{\textsc{\nouppercase{\leftmark}}}
\fancyfoot[R]{Page \thepage\ of \pageref{LastPage}}
\fancyfoot[C]{}

\begin{document}

\pagestyle{empty}
\begin{center}
{\Large \textbf{M3P17 Algebraic Combinatorics}}
\end{center}
\renewcommand{\contentsname}{}
\vspace{-60pt}
\tableofcontents

\pagebreak

\pagestyle{fancy}
\setcounter{page}{1}
\setcounter{section}{-1}
\setcounter{secnumdepth}{1}
\subsectionfont{\underline}

\section{Introduction}

Combinatorics in the study of discrete structures. These include:
\noparskip
\begin{points}
\item codes (subsets of $\Z_2^n$, where $\Z_2 = \set{0, 1}$),
\item graphs (vertices and edges),
\item designs (collection of subsets of a given set).
\end{points}

\subsection{Codes}

Aims of coding theory: To find codes $C$ such that:
\noparskip
\begin{points}
\item $C$ has many codewords,
\item $C$ corrects enough errors,
\item the length of $C$ is not too big.
\end{points}

\subsection{Graphs}

\begin{definition}
A graph is a pair $(V, E)$ where $V$ is a set of vertices, and $E$ is a collection of pairs $\set{x, y}$ (where $x, y \in V$) called edges.
\end{definition}

\begin{example}
If $V = \set{1, 2, 3, 4}$, $E = \set{\set{1, 2}, \set{1, 3}, \set{2, 3}, \set{2, 4}}$, then the graph is \pic[0.3]{1.png}.
\end{example}

\begin{definition}
For a vertex $x$, call the other vertices joined to $x$ by an edge the neighbours of $x$. Call $\Gamma$ a regular graph if every vertex has the same number of neighbours (say, $k$), and call $k$ the valency of $\Gamma$.
\end{definition}

\begin{example}
\begin{points}
\item \pic[0.15]{2.png} is regular with valency 2.
\item \pic[0.15]{3.png} is regular with valency 3.
\end{points}
\end{example}

\begin{definition}
A graph $\Gamma$ is strongly regular if:
\noparskip
\begin{points}
\item $\Gamma$ is regular with valency $k$,
\item any pair of joined vertices has the same number of common neighbours $a$,
\item any pair of non-joined vertices has the same number of common neighbours $b$.
\end{points}
\end{definition}

\begin{example}
\begin{points}
\item $\square$ is strongly regular, with $k = 2$, $a = 0$, $b = 2$.
\item The Petersen graph \pic[0.15]{4.png} is strongly regular, with $k = 3$, $a = 0$, $b = 1$.
\end{points}
\end{example}

\begin{result}{Proposition 0.1 (Friendship Thorem)}
In a community where any 2 people have exactly 1 common acquaintance, there is someone who knows everyone.
\end{result}

\begin{proof}{Proposition 0.1}
Let vertices = people, and join 2 vertices iff they know each other. Since every 2 vertices have exactly 1 common neighbour, the graph must look like \pic[0.15]{5.png} ie. a windmill (all known proofs use linear algebra).
\end{proof}

\subsection{Designs}

Suppose we have $v$ varieties of chocolate to be tested by consumers. We want each customer to test $k$ varieties, and each variety to be tested by $r$ consumers.

\begin{example}
Let $v = 8$, $k = 4$, $r = 3$, then number of consumers = $\frac{vr}{k} = 6$. Call the consumers $c_1$, ..., $c_6$, then $c_1$ tests 1234, $c_2$ tests 5678, $c_3$ tests 1357, $c_4$ tests 2468, $c_5$ tests 1247, $c_6$ tests 3568.
\end{example}

\begin{definition}
Let $X$ be a set, $v = \sizeof{X}$, $\cB$ be a collection of subsets of $X$. Call $(X, \cB)$ (or just $\cB$) a design if:
\noparskip
\begin{points}
\item every set in $\cB$ has size $k$,
\item every element of $X$ lies in $r$ subsets of $\cB$.
\end{points}
\end{definition}

The subsets in $\cB$ are called the blocks of the design, and the parameters of the design are $(v, k, r)$.

\begin{example}
The example $(8, 4, 3)$ above is a design.
\end{example}

\begin{definition}
A design $(X, \cB)$ is a 2-design if any 2 points (elements of $X$) lie in the same number of blocks.
\end{definition}

\begin{example}
The example $(8, 4, 3)$ above is not a 2-design.
\end{example}

In general, for $t \ge 1$, call $\cB$ a $t$-design if any $t$ points lie in the same number of blocks.

The larger $t$ is, the stronger the condition is. For large $t$, non-trivial $t$-designs are rare (in fact, the 1st non-trivial 6-design was found only in the 1980s).

\begin{example}
Let $p$ be a prime, then $\Z_p$ is a field. Call $\Z_p^2 = \set{(x_1, x_2): x_i \in \Z_p}$ the affine plane over $\Z_p$. Define a line in $\Z_p^2$ to be a subset of the form $\set{a + \lambda b: \lambda \in \Z_p}$, where $a$ and $b$ are fixed vectors in $\Z_p^2$, then any 2 vectors in $\Z_p^2$ lie on a unique line. Now let $X = \Z_p^2$, $\cB$ = collection of lines, then $(X, \cB)$ is a 2-design with parameters $(p^2, p, p + 1)$ (because there are $p + 1$ choices for $b$ and $p$ choices for the corresponding $a\implies r = \frac{kp (p + 1)}{v} = p + 1$).
\end{example}

\pagebreak

\section{Error-correcting Codes}

Define $\Z_2 = \set{0, 1}$, with addition and multiplication mod 2, and $\Z_2^n = \set{(x_1, \cdots, x_n): x_i \in \Z_2}$. With the usual addition and scalar multiplication, $\Z_2^n$ is a vector space over $\Z_2$, with standard basis $e_1, \cdots, e_n$ (where $e_k = \underbrace{0 \cdots 01}_k 0 \cdots 0$) and dimension $n$.

\begin{definition}
A code $C$ of length $n$ is a subset of $\Z_2^n$. The vectors in $C$ are called codewords, and the distance between 2 vectors in $\Z_2^n$ is $d(x, y)$ = number of coordinates where $x$ and $y$ differ.
\end{definition}

\begin{example}
$d(10111, 01110) = 3$.
\end{example}

\begin{result}{Proposition 1.1 (Triangle Inequality)}
$d(x, y) + d(y, z) \ge d(x, z)$.
\end{result}

\begin{proof}{Proposition 1.1}
Let $A = \set{i: x_i \ne z_i}$, $B = \set{i: x_i = y_i, x_i \ne z_i}$, $C = \set{i: x_i \ne y_i, x_i \ne z_i}$, then $\sizeof{A} = \sizeof{B} + \sizeof{C}$, $d(x, z) = \sizeof{A}$, $d(x, y) \ge \sizeof{C}$ and $d(y, z) \ge \sizeof{B} \implies d(x, y) + d(y, z) \ge \sizeof{C} + \sizeof{B} = \sizeof{A} = d(x, z)$.
\end{proof}

\begin{definition}
Let $C \subseteq \Z_2^n$ be a code. The minimum distance of $C$ is $d(C) = \min \set{d(x, y): x, y \in C, x \ne y}$.
\end{definition}

\begin{remark}
Let $C \subseteq \Z_2^n$, $e \in \N$, then we say $C$ corrects $e$ errors if whenever a codeword $c \in C$ is sent, and $\le e$ errors are made such that the vector $w$ is received, the closest codeword to $w$ is $c$.
\end{remark}

\begin{definition}
$C \subseteq \Z_2^n$ corrects $e$ errors if $\forall c_1, c_2 \in C$ and $w \in \Z_2^n, d(c_1, w), d(c_2, w) \le e \implies c_1 = c_2$.
\end{definition}

\begin{remark}
Equivalently, for $c \in C$, define a sphere $S_e(c) = \set{w \in \Z_2^n: d(c, w) \le e}$, then $C$ corrects $e$ errors if $S_e(c_1) \cap S_e(c_2) = \varnothing\ \forall c_1, c_2 \in C$, $c_1 \ne c_2$.
\end{remark}

\begin{result}{Proposition 1.2}
Code $C$ corrects $e$ errors iff $d(C) \ge 2e + 1$.
\end{result}

\begin{proof}{Proposition 1.2}
Suppose $d(C) \ge 2e + 1$. Pick $x, y \in C$, then if $w \in \Z_2^n$ satisfies $d(x, w), d(y, w) \le e$, by Proposition 1.1, $d(x, y) \le d(x, w) + d(y, w) \le 2e \implies x = y \implies C$ corrects $e$ errors. \\
Conversely, pick $x, y \in C$ such that $x \ne y$, $d(x, y) \le 2e$. Let $x, y$ possibly differ at bits $b_1, \cdots, b_{2e}$. Pick $w \in \Z_2^n$, such that $w_{b_i} = x_{b_i}$ for $1 \le i \le e$, $w_{b_i} = y_{b_i}$ for $e + 1 \le i \le 2e$, and $w_i = x_i = y_i$ everywhere else, then $d(x, w), d(y, w) \le e$ but $x \ne y \implies C$ does not correct $e$ errors.
\end{proof}

\subsection{Linear codes}

\begin{definition}
A linear code is a code $C \subseteq \Z_2^n$ which is a subspace of $\Z_2^n$ ie. $0 \in C$ and $x, y \in C \implies x + y \in C$.
\end{definition}

\begin{result}{Proposition 1.3}
Let $A$ be a $m \times n$ matrix over $\Z_2$. Then $C = \set{x \in \Z_2^n: Ax = 0}$ is a linear code, and $\dim C = n - \rank A$.
\end{result}

\begin{proof}{Proposition 1.3}
Easy peasy.
\end{proof}

\begin{example}
$C_3 = \set{abcxyz \in \Z_2^6: x = a + b, y = b + c, z = c + a} = \set{x \in \Z_2^6: \begin{pmatrix}
1 & 1 & 0 & 1 & 0 & 0 \\
0 & 1 & 1 & 0 & 1 & 0 \\
1 & 0 & 1 & 0 & 0 & 1 \\
\end{pmatrix} x = 0}$ is a linear code of dimension 3, with basis $\set{100101, 010110, 001011}$.
\end{example}

\begin{result}{Proposition 1.4}
If $C$ is a linear code with $\dim C = k$, then $\sizeof{C} = 2^k$.
\end{result}

\begin{proof}{Proposition 1.4}
Let $c_1, \cdots, c_k$ be a basis of $C$, then every $c \in C$ is a unique linear combination $c = \lambda_1 c_1 + \cdots + \lambda_k c_k$ where $\lambda_i \in \Z_2 \implies \sizeof{C} = \prod_i{(\text{number of choices for }\lambda_i)} = 2^k$.
\end{proof}

\subsection{Minimum distance}

\begin{definition}
For $x \in \Z_2^n$, the weight of $x$ is $\wt{x}$ = number of coordinates of $x$ equal to 1.
\end{definition}

\begin{remark}
$\wt{x} = d(x, 0)$, and $\wt{x + y} = d(x, y)$.
\end{remark}

\begin{result}{Proposition 1.5}
Let $C$ be a linear code, then $d(C) = \min \set{\wt{c}: c \in C \setminus \set{0}}$.
\end{result}

\begin{proof}{Proposition 1.5}
Let $c \in C \setminus \set{0}$ have minimal weight $r$. Since $C$ is linear, $0 \in C$ and $d(c, 0) = \wt{c} = r \implies d(C) \le r$. Now let $x, y \in C$ and $x \ne y$, then $x + y \in C \setminus \set{0} \implies d(x, y) = \wt{x + y} \ge r \implies d(C) \ge r$. Hence $d(C) = r$.
\end{proof}

\begin{example}
Consider $C_3 \subseteq \Z_2^6$. Check that $\min \set{\wt{c}: c \in C \setminus \set{0}} = 3$, hence $d(C_3) = 3 \implies C_3$ corrects 1 error by Proposition 1.2.
\end{example}

\subsection{Check matrices}

\begin{definition}
Suppose $A$ is a $m \times n$ matrix over $\Z_2$ and $C = \set{x \in \Z_2^n: Ax = 0}$. Then we call $A$ a check matrix of the linear code $C$.
\end{definition}

\begin{result}{Proposition 1.6}
Suppose the check matrix $A$ of the linear code $C$ satisfies:
\noparskip
\begin{points}
\item $A$ has no zero column,
\item $A$ does not have 2 equal columns.
\end{points}
\noparskip
Then $C$ corrects 1 error.
\end{result}

\begin{proof}{Proposition 1.6}
Suppose $C$ does not correct 1 error, then $d(C) \le 2$ by Proposition 1.2 $\implies$ by Proposition 1.5, $\exists c \in C \setminus \set{0}$ such that $\wt{c}$ = 1 or 2. If $\wt{c} = 1$, then $c = e_i \implies$ if $Ac = 0$, then the $i$-th column of $A$ is 0 ($\contradiction$). If $\wt{c} = 2$, then $c = e_i + e_j \implies$ if $Ac = 0$, then $Ae_i + Ae_j = 0 \implies$ the $i$-th and $j$-th column of $A$ are equal ($\contradiction$) $\implies C$ corrects 1 error.
\end{proof}

\begin{example}
\begin{points}
\item $C_3 = \set{x \in \Z_2^6: \begin{pmatrix}
1 & 1 & 0 & 1 & 0 & 0 \\
0 & 1 & 1 & 0 & 1 & 0 \\
1 & 0 & 1 & 0 & 0 & 1 \\
\end{pmatrix} x = 0}$ corrects 1 error by Proposition 1.6.
\item Suppose a code $C$ corrects 1 error and has a $3 \times n$ check matrix. By Proposition 1.6, to find the maximum dimension of $C$, we need to find the largest $n$ such that $\exists\ 3 \times n$ matrix with distinct non-zero columns in $\Z_2^3 \implies n = 2^3 - 1 = 7$. Pick $A = \begin{pmatrix}
1 & 1 & 1 & 0 & 1 & 0 & 0 \\
1 & 1 & 0 & 1 & 0 & 1 & 0 \\
1 & 0 & 1 & 1 & 0 & 0 & 1 \\
\end{pmatrix}$, then $C$ has dimension 4, and can send 16 messages $abcd$ using codewords $abcdxyz$, where $x = a + b + c$, $y = a + b + d$ and $z = a + c + d$. This is called a Hamming code, denoted $\Ham{3}$.
\end{points}
\end{example}

\subsection{Hamming codes}

\begin{definition}
Let $k \ge 3$, then a Hamming code $\Ham{k}$ is a code for which the check matrix has all the non-zero vectors in $\Z_2^k$ as columns.
\end{definition}

\begin{result}{Proposition 1.7}
\begin{points}
\item $\Ham{k}$ has length $2^k - 1$ and dimension $2^k - 1 - k$.
\item $\Ham{k}$ corrects 1 error.
\end{points}
\end{result}

\begin{proof}{Proposition 1.7}
\begin{points}
\item Since there are $2^k - 1$ non-zero vectors in $\Z_2^k$, the check matrix of $\Ham{k}$ is $k \times (2^k - 1)$ and has rank $k \implies$ the result follows.
\item Follows easily from Proposition 1.6.
\end{points}
\end{proof}

\begin{definition}
Let $C, C' \subseteq \Z_2^n$ be codes. Call $C$ and $C'$ equivalent codes if there is a permutation of their coordinates which sends the codewords in $C$ bijectively to those in $C'$.
\end{definition}

\begin{example}
All Hamming codes $\Ham{k}$ are equivalent.
\end{example}

\subsection{Correcting 1 error}

Suppose we have a code $C$ correcting 1 error, with check matrix $A$. A codeword $c$ is sent, and 1 error is made, so that $c'$ is received. Since $c' = c + e_i$ for some $i$, $Ac' = A(c + e_i) = Ac + Ae_i = 0 + Ae_i = i$-th column of $A \implies$ the error occurred in the $i$-th entry of $c$.

\begin{example}
Let $C = \Ham{3}$. Suppose we receive $c' = \begin{pmatrix} 1 & 1 & 0 & 1 & 0 & 0 & 0 \end{pmatrix}^\top$, then $Ac' = \begin{pmatrix} 0 \\ 1 \\ 0 \end{pmatrix}$ = 6th column of $A \implies$ the corrected codeword is $c = 1101010$.
\end{example}

\subsection{Correcting $> 1$ error}

\begin{result}{Proposition 1.8}
Let $d \ge 2$, $C$ be a code with check matrix $A$. Then:
\noparskip
\begin{points}
\item $d(C) \ge d$ if every set of $d - 1$ columns of $A$ is linearly independent,
\item $d(C) = d$ if, in addition, $\exists$ a set of $d$ columns of $A$ that are linearly dependent.
\end{points}
\end{result}

\begin{proof}{Proposition 1.8}
\begin{points}
\item Suppose $d(C) \le d - 1$, then $\exists c \in C \setminus \set{0}$ with $\wt{c} = r \le d - 1 \implies c = e_{i_1} + \cdots + e_{i_r} \implies Ac = Ae_{i_1} + \cdots + Ae_{i_r}$ = (sum of columns $i_1, \cdots, i_r$ of $A$) = 0 $\implies$ these columns are linearly dependent ($\contradiction$) $\implies d(C) \ge d$.
\item Suppose columns $i_1, \cdots, i_d$ of $A$ are linearly dependent, in addition to (1). Let $\lambda_1 A_{i_1} + \cdots + \lambda_d A_{i_d} = 0$ for some $\lambda_r \in \Z_2$. Since any $d - 1$ columns of $A$ are linearly independent, we must have $\lambda_r = 1\ \forall r \implies A(e_{i_1} + \cdots + e_{i_d}) = 0 \implies$ write $c = e_{i_1} + \cdots + e_{i_d}$, then $c \in C$ and $\wt{c} = d \implies$ since $d(C) \ge d$ by (1), we must have $d(C) = d$.
\end{points}
\end{proof}

\begin{example}
If we want a linear code of length 9 and dimension 2 which corrects 2 errors, the check matrix $A$ should be $7 \times 9$ (of rank 7), and we also need $C = \set{x \in \Z_2^9: Ax = 0}$. By Proposition 1.8, to have $d(C) \ge 5$, we need every set of 4 columns of $A$ to be linearly independent. Take $A = \begin{pmatrix} c_1 & c_2 & I_7 \end{pmatrix}$, then we need $\wt{c_1}, \wt{c_2} \ge 4$, and $\wt{c_1 + c_2} \ge 3 \implies$ let $c_1 = \begin{pmatrix} 1 & 1 & 1 & 1 & 0 & 0 & 0 \end{pmatrix}^\top$, $c_2 = \begin{pmatrix} 0 & 0 & 0 & 1 & 1 & 1 & 1 \end{pmatrix}^\top$, then they define the code $C = \set{abaaa(a + b)bbb: a, b \in \Z_2} = \set{0^9, 101111000, 010001111, 111110111}$.
\end{example}

\subsection{Hamming bound}

\begin{result}{Proposition 1.9}
$\sizeof{S_e(v)} = 1 + \binom{n}{1} + \cdots + \binom{n}{e}$.
\end{result}

\begin{proof}{Proposition 1.9}
Let $d_i$ = number of $x \in \Z_2^n$ such that $d(v, x) = i$, then $\sizeof{S_e(v)} = d_0 + d_1 + \cdots + d_e$. The vectors with distance $i$ from $v$ are precisely those differing from $v$ at exactly $i$ coordinates $\implies d_i = \binom{n}{i} \implies$ the result follows.
\end{proof}

\begin{result}{Theorem 1.10 (Hamming bound)}
Let $C$ be a code of length $n$, correcting $e$ errors. Then $\sizeof{C} \le \frac{2^n}{1 + \binom{n}{1} + \cdots + \binom{n}{e}}$.
\end{result}

\begin{proof}{Theorem 1.10}
Since $C$ corrects $e$ errors, the spheres $S_e(c)$ for $c \in C$ are all disjoint $\implies \sizeof{\bigcup_{c \in C}{S_e(c)}} = \sizeof{C} \sizeof{S_e(c)}$. But $\bigcup_{c \in C}{S_e(c)} \subseteq \Z_2^n$, so $2^n \ge \sizeof{\bigcup_{c \in C}{S_e(c)}} = \sizeof{C} \left[1 + \binom{n}{1} + \cdots + \binom{n}{e} \right] \implies \sizeof{C} \le \frac{2^n}{1 + \binom{n}{1} + \cdots + \binom{n}{e}}$.
\end{proof}

\begin{example}
Let $C$ be a linear code of length 9 that corrects 2 errors, then by Theorem 1.10, $\sizeof{C} \le \frac{2^9}{1 + \binom{9}{1} + \binom{9}{2}} = \frac{2^9}{46} < 2^4 \implies \dim C \le 3$. From the previous example, $\exists C$ with $\dim C = 2$. To find if $\exists C$ with $\dim C = 3$, we need a $6 \times 9$ check matrix $A$ with any 4 columns linearly independent. Take $A = \begin{pmatrix} c_1 & c_2 & c_3 & I_6 \end{pmatrix}$, then $c_1, c_2, c_3$ satisfy $\wt{c_i} \ge 4\ \forall i$, $\wt{c_i + c_j} \ge 3\ \forall i \ne j$, and $\wt{c_1 + c_2 + c_3} \ge 2$. After a tedious exercise, it can be shown that $\nexists c_i \implies \nexists C$.
\end{example}

\subsection{Perfect codes}

\begin{definition}
A code $C \in \Z_2^n$ is $e$-perfect ($e \ge 1$) if it corrects $e$ errors, and $\sizeof{C} = \frac{2^n}{1 + \binom{n}{1} + \cdots + \binom{n}{e}}$.
\end{definition}

\begin{remark}
Equivalently, the union of all the (disjoint) spheres $S_e(c)$ for $c \in C$ is the whole of $\Z_2^n$.
\end{remark}

\begin{result}{Proposition 1.11}
Let $C = \Z_2^n$, then $\sizeof{C} = \frac{2^n}{1 + n} \iff n = 2^k - 1, \sizeof{C} = 2^{n - k}$ for some $k$.
\end{result}

\begin{proof}{Proposition 1.11}
If $\sizeof{C} = \frac{2^n}{1 + n}$, then $1 + n = 2^k$ for some $k$. \\
Conversely, if $n = 2^k - 1$ and $\sizeof{C} = 2^{n - k}$, then obviously $\sizeof{C} = \frac{2^n}{1 + n}$.
\end{proof}

\begin{result}{Proposition 1.12}
$\Ham{k}$ is a 1-perfect code.
\end{result}

\begin{proof}{Proposition 1.12}
$\Ham{k}$ has length $n = 2^k - 1$, dimension $n - k$ and corrects 1 error $\implies \sizeof{\Ham{k}} = 2^{n - k} = \frac{2^n}{1 + n} \implies$ the result follows.
\end{proof}

\begin{remark}
The only $e$-perfect codes are:
\noparskip
\begin{points}
\item $\Ham{k}$, with $e = 1$,
\item $C = \set{0 \cdots 0, 1 \cdots 1}$, with length $n = 2e + 1$ and dimension 1,
\item the Golay code $G_{23}$, with $n = 23$, $e = 3$, $\dim G_{23} = 12$.
\end{points}
\end{remark}

\subsection{Gilbert-Varshamov bound}

\begin{example}
Let $C$ be a linear code of length 15, correcting 2 errors. Then the Hamming bound gives $\sizeof{C} \le \frac{2^{15}}{1 + \binom{15}{1} + \binom{15}{2}} = \frac{2^{15}}{121} < 2^9 \implies \dim C \le 8$.
\end{example}

\begin{result}{Theorem 1.13 (GV bound)}
Let $n, k, d \in \Z^+$ such that $1 + \binom{n - 1}{1} + \cdots + \binom{n - 1}{d - 2} < 2^{n - k}$, then $\exists$ a linear code of length $n$ and dimension $k$, such that $d(C) \ge d$.
\end{result}

\begin{example}
Let $n = 15$ and $d = 5$, then we have $1 + \binom{14}{1} + \binom{14}{2} + \binom{14}{3} = 470 < 2^9 = 2^{15 - 6} \implies \exists$ a code of dimension 6, but we still do not know if $\exists$ codes of dimension 7 or 8.
\end{example}

\begin{proof}{Theorem 1.13}
Assume $1 + \binom{n - 1}{1} + \cdots + \binom{n - 1}{d - 2} < 2^{n - k}$. We want to construct a check matrix $A$ such that $A$ is $(n - k) \times n$ (of rank $n - k$), and any $d - 1$ columns of $A$ are linearly independent. Choose the 1st $n-k$ columns of $A$ to be $e_1, \cdots, e_{n - k}$, then clearly they are linearly independent. Now suppose inductively that there are $i$ columns $c_1, \cdots, c_i \in \Z_2^{n - k}$ where $n - k \le i \le n - 1$, such that any $d - 1$ of these are linearly independent. The number of vectors in $\Z_2^{n - k}$ which are the sum of $\le d - 2$ of $c_1, \cdots, c_i$ is $\le 1 + \binom{i}{1} + \cdots + \binom{i}{d - 2} \le 1 + \binom{n - 1}{1} + \cdots + \binom{n - 1}{d - 2} < 2^{n - k}$, so $\exists c_{i + 1} \in \Z_2^{n - k}$ which is not the sum of $\le d - 2$ of $c_1, \cdots, c_i \implies$ if we have $A_i = \begin{pmatrix} c_1 & \cdots & c_i \end{pmatrix}$, we can extend it to get $A_{i + 1} = \begin{pmatrix} c_1 & \cdots & c_{i + 1} \end{pmatrix} \implies$ repeat until we get $A = A_n$ that satisfies all the required properties.
\end{proof}

\subsection{The Golay code}

The Golay code is a code of length 23, dimension 12, which corrects 3 errors and is perfect. To construct it, we first construct the extended Golay code $G_{24}$. Start with $H = \Ham{3}$, with check matrix $\begin{pmatrix}
1 & 1 & 1 & 0 & 1 & 0 & 0 \\
1 & 1 & 0 & 1 & 0 & 1 & 0 \\
1 & 0 & 1 & 1 & 0 & 0 & 1 \\
\end{pmatrix}$, and its reverse $K$, with check matrix $\begin{pmatrix}
0 & 0 & 1 & 0 & 1 & 1 & 1 \\
0 & 1 & 0 & 1 & 0 & 1 & 1 \\
1 & 0 & 0 & 1 & 1 & 0 & 1 \\
\end{pmatrix}$. Add the parity check bit (= sum of bits) to $H$ and $K$ to obtain $H'$ and $K'$ respectively, then we get $H' = \begin{Bmatrix}
00000000 & 11111111 \\
10001110 & 01110001 \\
01001101 & 10110010 \\
00101011 & 11010100 \\
00010111 & 11101000 \\
11000011 & 00111100 \\
10100101 & 01011010 \\
10011001 & 01100110 \\
\end{Bmatrix}$, $K' = \begin{Bmatrix}
00000000 & 11111111 \\
11100010 & 00011101 \\
01100101 & 10011010 \\
10101001 & 01010110 \\
11010001 & 00101110 \\
10000111 & 01111000 \\
01001011 & 10110100 \\
00110011 & 11001100 \\
\end{Bmatrix}$, both of which are linear codes of length 8 and dimension 4, with codewords of weight 0, 4 or 8. Also, the 14 codewords in $H'$ of weight 4 form a design with parameters $(16, 8, 7)$ ($v$ = number of bits $\times$ number of choices per bit = $8 \times 2 = 16$).

\begin{result}{Proposition 1.14}
$H \cap K = \set{0^7, 1^7}$, and $H' \cap K' = \set{0^8, 1^8}$.
\end{result}

\begin{proof}{Proposition 1.14}
Let $v \in H \cap K$, then $v = abcd(a + b + c)(a + b + d)(a + c + d)$ since $v \in H \implies$ since $v \in K$ too, we have $c + (a + b + c) + (a + b + d) + (a + c + d) = b + d + (a + b + d) + (a + c + d) = a + d + (a + b + c) + (a + c + d) = 0 \implies a + c = c + d = a + b = 0 \implies a = b = c = d$ = 0 or 1 $\implies v = 0^7$ or $1^7$. Also, by considering parity check bits, $H' \cap K' = \set{0^8, 1^8}$.
\end{proof}

\begin{definition}
The extended Golay code $G_{24}$ consists of all vectors in $\Z_2^{24}$ of the form $(a + x, b + x, a + b + x)$, where $a, b \in H'$, $x \in K'$.
\end{definition}

\begin{example}
\begin{points}
\item $a = b = x = 0^8 \implies v = 0^{24}$.
\item $a = b = x = 1^8 \implies v = (0^8, 0^8, 1^8)$.
\item $a = x = 1^8$, $b = 0^8 \implies v = (0^8, 1^8, 0^8)$.
\item $a = b = 0^8$, $x = 1^8 \implies v = 1^{24}$.
\item $a = 10001110$, $b = 10011001$, $x = 01001011 \implies v = 110001011101001001011100$.
\end{points}
\end{example}

\begin{result}{Proposition 1.15}
$G_{24}$ is a linear code of dimension 12.
\end{result}

\begin{proof}{Proposition 1.15}
Clearly $0^{24} \in G_{24}$. Now suppose $a_1, a_2, b_1, b_2 \in H'$, $x_1, x_2 \in K'$, then $(a_1 + x_1, b_1 + x_1, a_1 + b_1 + x_1) + (a_2 + x_2, b_2 + x_2, a_2 + b_2 + x_2) = (a_1 + a_2 + x_1 + x_2, b_1 + b_2 + x_1 + x_2, a_1 + a_2 + b_1 + b_2 + x_1 + x_2) \in G_{24}$ since $a_1 + a_2, b_1 + b_2 \in H'$ and $x_1 + x_2 \in K' \implies G_{24}$ is a linear code. \\
Moreover, $(a_1 + x_1, b_1 + x_1, a_1 + b_1 + x_1) = (a_2 + x_2, b_2 + x_2, a_2 + b_2 + x_2) \implies a_1 = a_2$, $b_1 = b_2$, $x_1 = x_2 \implies$ distinct choices of $(a, b, x)$ gives distinct elements of $G_{24} \implies \sizeof{G_{24}}$ = number of triples $(a, b, x) = \sizeof{H'}^2 \sizeof{K'} = 2^{12} \implies \dim \sizeof{G_{24}} = 12$.
\end{proof}

\begin{remark}
$(a + x, b + x, a + b + x) = (a, 0, a) + (0, b, b) + (x, x, x) \implies$ if $a_i$, $b_i$ and $x_i\ (1 \le i \le 4)$ are bases for $H'$, $H'$ and $K'$ respectively, then $\set{(a_i, 0, a_i), (0, b_i, b_i), (x_i, x_i, x_i)}$ form a basis for $G_{24}$.
\end{remark}

\begin{definition}
For $v, w \in \Z_2^n$, let $[v, w]$ = number of places where both $v$ and $w$ are 1.
\end{definition}

\begin{result}{Proposition 1.16}
Let $v, w \in \Z_2^n$, then:
\noparskip
\begin{points}
\item $\wt{v + w} = \wt{v} + \wt{w} - 2[v, w]$,
\item if $4 \mid \wt{v}$ and $4 \mid \wt{w}$, then $4 \mid \wt{v + w}$ iff $[v, w]$ is even.
\end{points}
\end{result}

\begin{proof}{Proposition 1.16}
\begin{points}
\item Let $r = \wt{v}$, $s = \wt{w}$ and $t = [v, w]$, then we have (reordering coordinates if required) $v = \underbrace{1 \cdots 1}_t \underbrace{1 \cdots 1}_{r - t} \underbrace{0 \cdots 0}_{s - t} 0 \cdots 0$, $w = \underbrace{1 \cdots 1}_t \underbrace{0 \cdots 0}_{r - t} \underbrace{1 \cdots 1}_{s - t} 0 \cdots 0 \implies v + w = \underbrace{0 \cdots 0}_t \underbrace{1 \cdots 1}_{r - t} \underbrace{1 \cdots 1}_{s - t} 0 \cdots 0 \implies \wt{v + w} = r + s - 2t = \wt{v} + \wt{w} - 2[v, w]$.
\item Follows easily from (1).
\end{points}
\end{proof}

\begin{result}{Proposition 1.17}
If $a, b, x \in \Z_2^n$, then $[a, x] + [b, x] + [a + b, x]$ is even.
\end{result}

\begin{proof}{Proposition 1.17}
Let $r = [a, x]$, $s = [b, x]$, $u$ = number of places where $a, b, x$ are all 1. Then (reordering coordinates if needed) $x = \underbrace{1 \cdots 1}_u \underbrace{1 \cdots 1}_{r - u} \underbrace{1 \cdots 1}_{s - u} 0 \cdots 0$, $a = \underbrace{1 \cdots 1}_u \underbrace{1 \cdots 1}_{r - u} \underbrace{0 \cdots 0}_{s - u} 0 \cdots 0$, $b = \underbrace{1 \cdots 1}_u \underbrace{0 \cdots 0}_{r - u} \underbrace{1 \cdots 1}_{s - u} 0 \cdots 0 \implies a + b = \underbrace{0 \cdots 0}_u \underbrace{1 \cdots 1}_{r - u} \underbrace{1 \cdots 1}_{s - u} 0 \cdots 0 \implies [a, x] + [b, x] + [a + b, x] = r + s + (r + s - 2u) = 2(r + s - u)$, which is even.
\end{proof}

\begin{result}{Proposition 1.18}
If $c \in G_{24}$, then $4 \mid \wt{c}$.
\end{result}

\begin{proof}{Proposition 1.18}
Let $c = (a + x, b + x, a + b + x)$ for some $a,b \in H'$, $x \in K'$, then $c = (a, b, a + b) + (x, x, x)$. Let $v = (a, b, a + b)$, $w = (x, x, x)$, then $4 \mid \wt{v}, \wt{w}$ since $4 \mid \wt{a}, \wt{b}, \wt{a + b}, \wt{x}$, and $[v, w] = [a, x] + [b, x] + [a + b, x]$ is even by Proposition 1.17 $\implies 4 \mid \wt{v + w} = \wt{c}$ by Proposition 1.16(2).
\end{proof}

\begin{result}{Theorem 1.19}
$d(G_{24}) = 8$.
\end{result}

\begin{proof}{Theorem 1.19}
Suppose $d(G_{24}) < 8$, then by Proposition 1.18, $\exists c \in G_{24} \setminus \set{0}$ such that $\wt{c} = 4$. Let $c = (a + x, b + x, a + b + x)$ for some $a,b \in H'$, $x \in K'$, then $\wt{a + x} = \wt{a} + \wt{x} - 2[a, x]$ is even. Similarly, $\wt{b + x}$ and $\wt{a + b + x}$ are all even $\implies\ \ge 1$ of $a + x$, $b + x$, $a + b + x$ must be 0 $\implies x = a, b$ or $a + b \implies x \in H' \cap K' = \set{0^8, 1^8}$ by Proposition 1.14 $\implies a + x, b + x, a + b + x \in H' \implies a + x, b + x, a + b + x$ have weight 0, 4 or 8 $\implies$ 2 of these are $0^8$. If $a + x = b + x = 0^8$, then $a = x = b \implies c = (0^8, 0^8, x)$. If $a + x = a + b + x = 0^8$, then $a = x$, $b = 0^8 \implies c = (0^8, x, 0^8)$. If $b + x = a + b + x = 0^8$, then $b = x$, $a = 0^8 \implies c = (x, 0^8, 0^8)$. Either way, $\wt{c}$ = 0 or 8 $(\contradiction) \implies d(G_{24}) \ge 8 \implies$ since $(1^8, 0^8, 0^8) \in G_{24}$, $d(G_{24}) = 8$.
\end{proof}

\subsection{The 3-perfect code $G_{23}$}

\begin{definition}
The Golay code $G_{23}$ is the code of length 23 consisting of codewords in $G_{24}$ with the last bit deleted.
\end{definition}

\begin{remark}
$G_{23}$ is linear, and $\sizeof{G_{23}} = \sizeof{G_{24}} = 2^{12} \implies \dim G_{23} = 12$.
\end{remark}

\begin{result}{Theorem 1.20}
$G_{23}$ is 3-perfect.
\end{result}

\begin{proof}{Theorem 1.20}
$d(G_{24}) = 8 \implies d(G_{23}) \ge 7$, and $(0^8, 0^8, 1^8) \in G_{24} \implies d(G_{23}) = 7 \implies G_{23}$ corrects 3 errors. Also, $\sizeof{G_{23}} = \frac{2^{23}}{1 + \binom{23}{1} + \binom{23}{2} + \binom{23}{3}} = \frac{2^{23}}{2048} = 2^{12} \implies G_{23}$ is 3-perfect. 
\end{proof}

\begin{remark}
Codewords in $G_{24}$ are those in $G_{23}$ with parity check bit added.
\end{remark}

\subsection{A 5-design from $G_{24}$}

Define $X$ = set of 24 coordinate positions in $G_{24}$, and a block $B_c$ = set of 8 coordinate positions of the 1's in each codeword $c \in G_{24}$ of weight 8. Call the blocks the octads of $G_{24}$.

\begin{result}{Theorem 1.21}
The octads of $G_{24}$ form the blocks of a 5-design, where every set of 5 points lies in a unique octad.
\end{result}

\begin{proof}{Theorem 1.21}
There is a correspondence $\Z_2^{24} \leftrightarrow$ subsets of $X$, $v \leftrightarrow P_v$ = set of positions of 1's in $v$. Let $v \in \Z_2^{24}$ have weight 5, and delete the last bit of $v$ to get $v' \in \Z_2^{23}$, with $\wt{v'}$ = 4 or 5, $P_{v'} \subseteq \set{1, \cdots, 23}$. Since $G_{23}$ is 3-perfect, $\exists! c' \in G_{23}$ such that $v' \in S_3(c')$ ie. $d(v', c') \le 3$. \\
If $\wt{v'} = 4$, then $\wt{c'} = 7$, and $P_{v'} \subseteq P_{c'}$. Add the parity check bit of $c'$ to get $c \in G_{24}$, with $\wt{c} = 8 \implies P_v = P_{v'} \cup \set{24} \subseteq P_{c'} \cup \set{24} = P_c$. \\ Otherwise, if $\wt{v'} = 5$, then $\wt{c'}$ = 7 or 8, and $P_{v'} \subseteq P_{c'}$ too. Again, add the parity check bit of $c'$ to get $c \in G_{24}$, with $\wt{c} = 8 \implies P_v = P_{v'} \subseteq P_{c'} \subseteq P_c$. \\
Either way, $\exists! c \in G_{24}$ where $\wt{c} = 8$, with $P_v \subseteq P_c = B_c \implies$ the result follows.
\end{proof}

\begin{result}{Proposition 1.22}
\begin{points}
\item Codewords in $G_{24}$ have weight 0, 8, 12, 16 or 24, and $N_i = N_{24 - i}$, where $N_i$ is the number of codewords in $G_{24}$ with weight $i$.
\item Codewords in $G_{23}$ have weight 0, 7, 8, 11, 12, 15, 16 or 23, and $M_i = M_{23 - i}$, where $M_i$ is the number of codewords in $G_{23}$ with weight $i$.
\end{points}
\end{result}

\begin{proof}{Proposition 1.22}
\begin{points}
\item The map $G_{24} \rightarrow G_{24}$, $c \mapsto c + 1^{24}$ is a bijection that sends codewords of weight $i$ to codewords of weight $24 - i \implies N_i = N_{24 - i}$. Also, pick $c \in G_{24} \setminus \set{0}$, then $4 \mid \wt{c}$ by Proposition 1.18 and $\wt{c} \ge 8$ by Theorem 1.19 $\implies \wt{c}$ = 8, 12, 16 or 24.
\item Similar to (1).
\end{points}
\end{proof}

\begin{result}{Proposition 1.23}
Let $X$ be a set of $v$ points, $\cB$ be a $t$-design with blocks of size $k$, in which any $t$ points lie in $r_t$ blocks. Then $\cB$ is a $(t - 1)$-design, and $r_{t - 1} = \left( \frac{v - t + 1}{k - t + 1} \right) r_t$.
\end{result}

\begin{proof}{Proposition 1.23}
Pick $S \subseteq X$, $\sizeof{S} = t - 1$, $r(S)$ = number of blocks containing $S$. Consider pairs $(x, B)$, where $x \in X \setminus S$ and $B$ is a block containing $S \cup \set{x}$, then the number of such pairs = ways to choose $x\ \times$ ways to choose $B$ given $x = (v - (t - 1)) \times r_t$. \\
On the other hand, the number of such pairs is also = ways to choose $B\ \times$ ways to choose $x$ given $B = r(S) \times (k - (t - 1)) \implies r(S) = \left( \frac{v - t + 1}{k - t + 1} \right) r_t \implies$ the result follows.
\end{proof}

\begin{result}{Corollary 1.24}
A $t$-design is also an $s$-design $\forall 1 \le s \le t$, and $r_{t - 2} = \left( \frac{v - t + 2}{k - t + 2} \right) r_{t - 1}$, $\cdots$, $r = r_1 = \left( \frac{v - 1}{k - 1} \right) r_2$, $b = r_0 = \frac{vr}{k}$.
\end{result}

\begin{proof}{Corollary 1.24}
Follows easily from Proposition 1.23.
\end{proof}

\begin{result}{Proposition 1.25}
\begin{points}
\item In $G_{24}$, $N_{16} = N_8$ = number of octads = 759.
\item In $G_{23}$, $M_7 = 253$, $M_8 = 506$.
\end{points}
\end{result}

\begin{proof}{Proposition 1.25}
\begin{points}
\item Applying Corollary 1.24 to the 5-design formed by the octads of $G_{24}$ gives $r_5 = 1$, $r_4 = \left( \frac{24 - 5 + 1}{8 - 5 + 1} \right) r_5 = 5$, $r_3 = \left( \frac{24 - 4 + 1}{8 - 4 + 1} \right) r_4 = 21$, $r_2 = \left( \frac{24 - 3 + 1}{8 - 3 + 1} \right) r_3 = 77$, $r_1 = \left( \frac{24 - 2 + 1}{8 - 2 + 1} \right) r_2 = 253$, $N_{16} = N_8 = r_0 = \left( \frac{24 - 1 + 1}{8 - 1 + 1} \right) r_1 = 759$.
\item $M_7$ = (number of octads containing point 24) = $r_1 = 253 \implies M_8 = N_8 - M_7 = 506$.
\end{points}
\end{proof}

\subsection{Error correction in $G_{24}$}

\begin{result}{Proposition 1.26}
$\forall c, d \in G_{24}$, $c \cdot d = c^\top d = 0 \in \Z_2$.
\end{result}

\begin{proof}{Proposition 1.26}
By Proposition 1.18, $4 \mid \wt{c}, \wt{d}, \wt{c + d}\ \forall c, d \in G_{24} \implies$ by Proposition 1.17, since $\wt{c + d} = \wt{c} + \wt{d} - 2[c, d]$, $[c, d]$ is even $\implies c^\top d = 0$.
\end{proof}

\begin{remark}
With a basis $\set{c_i: 1 \le i \le 12}$ of $G_{24}$, let $A = \begin{pmatrix} c_1 & \cdots & c_{12} \end{pmatrix}^\top$ with size $12 \times 24$, then $Ac = \begin{pmatrix} c_1 \cdot c & \cdots & c_{12} \cdot c \end{pmatrix}^\top = 0\ \forall c \in G_{24}$. Moreover, since $\dim G_{24} = 12$, $G_{24}$ is the solution space for $Ax = 0 \implies A$ is a check matrix for $G_{24}$.
\end{remark}

Suppose $c \in G_{24}$ is sent and $t \le 3$ errors are made, such that the received vector is $x = c + e_{i_1} + \cdots + e_{i_t}$. Let the 253 codewords in $G_{24}$ with weight 8 and a 1 in the 1st coordinate be $c_1, \cdots c_{253}$, with corresponding octads $B_1, \cdots B_{253}$, then $c_i \cdot x = 0$ for $1 \le i \le 253$ if $x \in G_{24}$, else we can count how many $c_i \cdot x = 1$ there are.

\begin{result}{Proposition 1.27}
The number of $i$ such that $c_i \cdot x = 1$ is: \\
$\begin{array}{c|cccc}
t & 1 & 2 & 3 & 4 \\ \hline
x_1 \text{ correct} & 77 & 112 & 125 & 128 \\
x_1 \text{ wrong} & 253 & 176 & 141 & 128
\end{array}$
\end{result}

\begin{proof}{Proposition 1.27}
When $t = 1$, $x = c + e_j$ for some $c \in G_{24} \implies c_i \cdot x = c_i \cdot (c + e_j) = c_i \cdot e_j = 1$ iff $j \in B_i$. 
If $x_1$ is correct, then $j \ne 1 \implies$ (number of $c_i \cdot x = 1$) = (number of $B_i$ containing $j$) = (number of octads containing 1 and $j$) = $r_2 = 77$. 
Otherwise, if $x_1$ is wrong, then $k = 1 \implies$ (number of $c_i \cdot x = 1$) = (number of $B_i$ containing 1) = $r_1 = 253$. \\
When $t = 2$, $x = c + e_j + e_k$ for some $c \in G_{24} \implies c_i \cdot x = c_i \cdot e_j + c_i \cdot e_k = 1$ iff exactly 1 of $j, k \in B_i$. 
If $x_1$ is correct, then $j, k \ne 1 \implies$ (number of $c_i \cdot x = 1$) = (number of octads containing 1 and $j$ but not $k$, or 1 and $k$ but not $j$) = $2(r_2 - r_3) = 2(77 - 21) = 112$. 
Otherwise, if $x_1$ is wrong, let $j = 1$ WLOG $\implies$ (number of $c_i \cdot x = 1$) = (number of $B_i$ containing 1 but not $k$) = $r_1 - r_2 = 253 - 77 = 176$. \\
When $t = 3$, $x = c + e_j + e_k + e_l$ for some $c \in G_{24} \implies c_i \cdot x = 1$ iff exactly 1 or 3 of $j, k, l \in B_i$. 
If $x_1$ is correct, then $j, k, l \ne 1 \implies$ (number of $c_i \cdot x = 1$) = $3(r_2 - 2r_3 + r_4) + r_4 = 3(77 - 42 + 5) + 5 = 125$. 
Otherwise, if $x_1$ is wrong, let $j = 1$ WLOG $\implies$ (number of $c_i \cdot x = 1$) = $(r_1 - 2r_2 + r_3) + r_3 = (253 - 154 + 21) + 21 = 141$. \\
When $t = 4$, $x = c + e_j + e_k + e_l + e_m$ for some $c \in G_{24} \implies c_i \cdot x = 1$ iff exactly 1 or 3 of $j, k, l, m \in B_i$. 
If $x_1$ is correct, then $j, k, l, m \ne 1 \implies$ (number of $c_i \cdot x = 1$) = $4(r_2 - 3r_3 + 3r_4 - r_5) + 4(r_4 - r_5) = 4(77 - 63 + 15 - 1) + 4(5 - 1) = 128$. 
Otherwise, if $x_1$ is wrong, let $j = 1$ WLOG $\implies$ (number of $c_i \cdot x = 1$) = $(r_1 - 3r_2 + 3r_3 - r_4) + 3(r_3 - r_4) = (253 - 231 + 63 - 5) + 3(21 - 5) = 128$.
\end{proof}

\subsection{Cyclic codes}

\begin{definition}
A linear code $C \in \Z_2^n$ is cyclic if $(c_1, \cdots, c_n) \in C \implies (c_n, c_1, \cdots, c_{n - 1}) \in C$.
\end{definition}

\begin{remark}
The definition implies that all other cyclic shifts are also $\in C$.
\end{remark}

\begin{example}
\begin{points}
\item $C = \set{000, 110, 101, 011} \subseteq \Z_2^3$ is cyclic.
\item $\Ham{3}$, with check matrix $A = \begin{pmatrix}
1 & 0 & 1 & 1 & 1 & 0 & 0 \\
1 & 1 & 1 & 0 & 0 & 1 & 0 \\
0 & 1 & 1 & 1 & 0 & 0 & 1 \\
\end{pmatrix}$ is cyclic, because the shifted check matrix $A' = \begin{pmatrix}
0 & 1 & 0 & 1 & 1 & 1 & 0 \\
0 & 1 & 1 & 1 & 0 & 0 & 1 \\
1 & 0 & 1 & 1 & 1 & 0 & 0 \\
\end{pmatrix}$ is in fact $\begin{pmatrix} A_1 + A_2 \\ A_3 \\ A_1 \\ \end{pmatrix}$, where $A_i = i$-th row of $A$.
\item $G_{23}$ is equivalent to a cyclic code.
\end{points}
\end{example}

\subsection{Ideals}

\begin{definition}
A commutative ring $(R, +, \times)$ is a set $R$ with $+, \times$ such that:
\noparskip
\begin{points}
\item $(R, +)$ is an abelian group with identity 0,
\item $(R, \times)$ is commutative and associative,
\item $\forall a, b, c \in R$, $a \times (b + c) = (a \times b) + (a \times c)$.
\end{points}
\end{definition}

\begin{example}
$\Ztwox$ is the ring of polynomials $a_0 + a_1 x + \cdots + a_n x^n$ with $a_i \in \Z_2$ and normal $+, \times$ for polynomials.
\end{example}

\begin{definition}
Let $R$ be a commutative ring, then a subset $I \subseteq R$ is an ideal if:
\noparskip
\begin{points}
\item $I$ is (am!) a subgroup of $(R, +)$,
\item $IR = \set{ir: i \in I, r \in R} \subseteq I$.
\end{points}
\end{definition}

\begin{example}
Let $a \in R$, and define $\gen{a} = \set{ar: r \in R}$, then $\gen{a}$ is an ideal, called the principal ideal generated by $a$. 
\end{example}

\subsection{Quotient rings}

Let $I$ be an ideal of $R$. For $x \in R$, define the coset $x + I = \set{x + i : i \in I}$, and call the set of all cosets $\frac{R}{I}$. Define $+, \times$ on $\frac{R}{I}$ by $(x + I) + (y + I) = (x + y) + I$, $(x + I)(y + I) = xy + I$, then they are well-defined, and make $\frac{R}{I}$ into a (commutative) ring, called the quotient ring.

\begin{example}
Consider $\Ztwoxover{I}$, where $I = \gen{x^2 - 1}$, then $\set{I, 1 + I, x + I, 1 + x + I} \subseteq \Ztwoxover{I}$. Now let $f(x) + I \in \Ztwoxover{I}$, then $f(x) = (x^2 - 1)q(x) + r(x)$, where $\deg r < 2 \implies f(x) + I = r(x) + (x^2 - 1)q(x) + I = r(x) + I$ since $(x^2 - 1)q(x) \in I \implies r(x)$ is either $1, x, 1 + x$ or 0 $\implies \Ztwoxover{I} = \set{I, 1 + I, x + I, 1 + x + I}$.
\end{example}

\begin{notation}
Write $x + I = \xbar$, then $\Ztwoxover{I} = \set{0, 1, \xbar, 1 + \xbar}$.
\end{notation}

\begin{result}{Proposition 1.28}
Let $R = \Ztwoxover{I}$ where $I = \gen{x^n - 1}$, $\xbar = x + I \in R$, then $R = \set{a_0 + \cdots + a_{n - 1} \xbar^{n - 1}: a_i \in \Z_2}$, with the usual addition and multiplication determined by the relation $\xbar^n = 1$.
\end{result}

\begin{proof}{Proposition 1.28}
Let $S = \set{a_0 + \cdots + a_{n - 1} \xbar^{n - 1}: a_i \in \Z_2}$, then clearly $S \subseteq R$. Now let $f(\xbar) \in R$, then $f(x) = (x^n - 1)q(x) + r(x)$, where $\deg r < n \implies f(\xbar) = r(\xbar) + (\xbar^n - 1)q(\xbar) = r(\xbar) \in S \implies R \subseteq S \implies R = S$.
\end{proof}

\begin{example}
Let $R = \Ztwoxgen{x^3 - 1}$, then $(1 + \xbar)(1 + \xbar^2) = 1 + \xbar + \xbar^2 + \xbar^3 = \xbar + \xbar^2$.
\end{example}

\begin{remark}
By Proposition 1.28, $\exists$ a bijection $\pi: \Z_2^n \rightarrow \Ztwoxgen{x^n - 1}$, $(a_0, \cdots, a_{n - 1}) \mapsto a_0 + \cdots + a_{n - 1} \xbar^{n - 1}$, which is also an isomorphism of groups under +.
\end{remark}

\begin{example}
Let $C = \set{000, 110, 011, 101} \subseteq \Z_2^3$, then $\pi(C) = \set{0, 1 + \xbar, \xbar + \xbar^2, 1 + \xbar^2} \subseteq \Ztwoxgen{x^3 - 1}$.
\end{example}

\begin{result}{Proposition 1.29}
$C \subseteq \Z_2^n$ is a cyclic (linear) code iff $\pi(C)$ is an ideal of $\Ztwoxgen{x^n - 1}$.
\end{result}

\begin{proof}{Proposition 1.29}
Suppose $\pi(C) = I$ is an ideal. Let $c, d \in C$, then $\pi(c), \pi(d) \in I \implies \pi(c + d) = \pi(c) + \pi(d) \in I \implies c + d \in C \implies C$ is a linear code. Now write $c = (c_0, \cdots, c_{n - 1}) \in C$, then $\pi(c) = c_0 + \cdots + c_{n - 1} \xbar^{n - 1} \in I \implies c_{n - 1} + c_0 \xbar + \cdots + c_{n - 1} \xbar^{n - 1} = c_{n - 1} \xbar^n + c_0 \xbar + \cdots + c_{n - 1} \xbar^{n - 1} = \xbar \pi(c) \in I \implies (c_{n - 1}, c_0, \cdots, c_{n - 2}) \in C \implies C$ is a cyclic code. \\
Conversely, suppose $C$ is a cyclic code, then $I = \pi(C)$ is a subgroup of $\Ztwoxgen{x^n - 1}$ since $C$ is linear and $0 = \pi(0^n) \in I$. Let $f(\xbar) = f_0 + \cdots + f_{n - 1} \xbar^{n - 1} \in I$, then $\pi^{-1}(f(\xbar)) = (f_0, \cdots, f_{n - 1}) \in C \implies (f_{n - 1}, f_0, \cdots, f_{n - 2}) \in C \implies \xbar f(\xbar) = f_0 \xbar + \cdots + f_{n - 1} \xbar^n = f_{n - 1} + f_0 \xbar + \cdots + f_{n - 2} \xbar^{n - 1} \in I$. Similarly, $\xbar^i f(\xbar) \in I\ \forall i \implies g(\xbar)f(\xbar) \in I\ \forall g(\xbar) \in \Ztwoxgen{x^n - 1} \implies I = \pi(C)$ is an ideal.
\end{proof}

\subsection{Basic construction of cyclic codes}

\begin{definition}
Fix $n \in \N$, let $p(x) \in \Ztwox$, $p(x) \mid x^n - 1$, $I$ be the ideal of $\Ztwoxgen{x^n - 1}$ defined by $I = \gen{p(\xbar)} = \set{p(\xbar)f(\xbar): f(\xbar) \in \Ztwoxgen{x^n - 1}}$. Then $p(x)$ is called a generator polynomial for the cyclic code $C = \pi^{-1}(I) \subseteq \Z_2^n$.
\end{definition}

\begin{example}
\begin{points}
\item Let $n = 3$, $p(x) = x + 1$, then $p(x) \mid x^3 - 1 \implies I = \gen{p(\xbar)} = \set{0, 1 + \xbar, 1 + \xbar^2, \xbar + \xbar^2} \implies$ the corresponding cyclic code is $C = \set{000, 110, 101, 011}$.
\item Let $n = 6$, then $x^6 - 1 = (x^3 + 1)^2 = (x + 1)^2 (x^2 + x + 1)^2$ in $\Ztwox \implies$ number of $p(x)$ dividing $x^6 - 1$ = (number of choices for $(x + 1)^i (x^2 + x + 1)^j$ where $0 \le i, j \le 2$) = $(2 + 1)(2 + 1) = 9$.
\item From (2), let $p(x) = (x^2 + x + 1)^2 = x^4 + x^2 + 1$, then $C = \pi^{-1}\left( \gen{\xbar^4 + \xbar^2 + 1} \right) = \set{000000, 101010, 010101, 111111}$.
\end{points}
\end{example}

\begin{result}{Proposition 1.30}
If $\deg p = n - k$, then $\dim C = k$.
\end{result}

\begin{proof}{Proposition 1.30}
It suffices to show that $S = \set{p(\xbar), \cdots, \xbar^{k - 1} p(\xbar)}$ is a basis for $\gen{p(\xbar)} = \pi(C)$ as a subspace of $\Ztwoxgen{x^n - 1}$ over $\Z_2$. Suppose $f(\xbar) = \sum_{i = 0}^{k - 1} \lambda_i \xbar^i p(\xbar) = 0$ in $\Ztwoxgen{x^n - 1}$ for some $\lambda_i \in \Z_2$, then $x^n - 1 \mid f(x) \implies f(x) = 0$ in $\Ztwox$ since $\deg f \le (n - k) + (k - 1) = n - 1 \implies$ by comparing coefficients, $\lambda_i = 0\ \forall i \implies S$ is a linearly independent set. \\
Now pick $h(\xbar) \in \gen{p(\xbar)}$, then $h(\xbar) = g(\xbar)p(\xbar)$ for some $g(\xbar) \in \Ztwoxgen{x^n - 1}$. Long division gives $g(x)p(x) = q(x)(x^n - 1) + r(x)$ where $\deg r < n \implies p(x) \mid q(x)(x^n - 1) + r(x) \implies p(x) \mid r(x)$. Let $r(x) = p(x)s(x)$, then $\deg s \le n - (n - k) = k \implies$ since $g(x)p(x) = q(x)(x^n - 1) + p(x)s(x)$, $h(\xbar) = g(\xbar)p(\xbar) = 0 + p(\xbar)s(\xbar) \implies h(\xbar)$ is a linear combination of elements in $S \implies \gen{p(\xbar)} \subseteq \span{S} \implies S$ is a basis for $\gen{p(\xbar)} = \pi(C)$.
\end{proof}

\begin{example}
Let $n = 7$, then $x^7 - 1 = (x + 1)(x^3 + x + 1)(x^3 + x^2 + 1)$ in $\Ztwox$. Pick $p(x) = x^3 + x + 1$ and let $C$ be its corresponding cyclic code, then $\dim C = 4$, and a basis of $C$ is $\set{1101000, 0110100, 0011010, 0001101}$.
\end{example}

\subsection{Check matrices for cyclic codes}

Let $p(x) = p_0 + \cdots + p_{n - k} x^{n - k} \mid x^n - 1$ be the generator polynomial for a cyclic code $C$, and call $G = \begin{pmatrix}
p_0 & \cdots & p_{n - k} &           &            0 \\
      & \ddots &               & \ddots &               \\
   0 &           &         p_0 & \cdots & p_{n - k} \\
\end{pmatrix}$ (a $k \times n$ matrix) the generator matrix of $C$.

\begin{result}{Proposition 1.31}
Let $q(x) = q_0 + \cdots + q_k x^k = \frac{x^n - 1}{p(x)}$ in $\Ztwox$, and $H = \begin{pmatrix}
   0 &            & q_k &  \cdots & q_0 \\
      & \iddots &       & \iddots &       \\
q_k &  \cdots & q_0 &           &     0 \\
\end{pmatrix}$ (a $(n - k) \times n$ matrix), then $H$ is a check matrix for $C$.
\end{result}

\begin{proof}{Proposition 1.31}
Let $q(x)p(x) = \sum_{d = 0}^n f_d x^d$, then $f_d = \sum q_i p_{d - i} = 0$ for $1 \le d \le n - 1$ since $q(x)p(x) = x^n - 1 \implies HG^\top = \begin{pmatrix}
   0 &            & q_k &  \cdots & q_0 \\
      & \iddots &       & \iddots &       \\
q_k &  \cdots & q_0 &           &     0 \\
\end{pmatrix} \begin{pmatrix}
        p_0 &           &           0 \\
     \vdots & \ddots &              \\
p_{n - k} &           &        p_0 \\
              & \ddots &     \vdots \\
           0 &           & p_{n - k} \\
\end{pmatrix} = \begin{pmatrix}
f_{n - 1} & \cdots & f_{n - k} \\
    \vdots & \ddots &    \vdots \\
        f_k & \cdots &         f_1 \\
\end{pmatrix} = 0$. Pick $c \in C$, then $c$ can be written as a linear combination of the rows of $G \implies Hc = \sum H$(some columns of $G^\top$) = $\sum 0 = 0 \implies H$ is a check matrix for $C$.
\end{proof}

\begin{example}
Let $n = 7$, then $x^7 - 1 = (x + 1)(x^3 + x + 1)(x^3 + x^2 + 1)$. Pick $p(x) = x^3 + x + 1$, then $G = \begin{pmatrix}
1 & 1 & 0 & 1 & 0 & 0 & 0 \\
0 & 1 & 1 & 0 & 1 & 0 & 0 \\
0 & 0 & 1 & 1 & 0 & 1 & 0 \\
0 & 0 & 0 & 1 & 1 & 0 & 1 \\
\end{pmatrix}$. Also, $q(x) = (x + 1)(x^3 + x^2 + 1) = x^4 + x^2 + x + 1 \implies H = \begin{pmatrix}
0 & 0 & 1 & 0 & 1 & 1 & 1 \\
0 & 1 & 0 & 1 & 1 & 1 & 0 \\
1 & 0 & 1 & 1 & 1 & 0 & 0 \\
\end{pmatrix}$ is a check matrix for the cyclic code $C$ generated by $p(x) \implies C = \Ham{3}$.
\end{example}

\subsection{BCH codes}

It is usually hard to tell what $d(C)$ of a cyclic code $C$ is, but some special cyclic codes allow $d(C)$ to be computed.

\begin{definition}
A polynomial $f(x) \in \Ztwox$ with $\deg f \ge 1$ is irreducible if it cannot be factorized as a product of polynomials in $\Ztwox$ of smaller degree.
\end{definition}

\begin{example}
\begin{points}
\item $x, x + 1$ are irreducible.
\item $x^2 + 1 = (x + 1)^2$ is reducible but $x^2 + x + 1$ is irreducible (no root in $\Z_2$).
\item The irreducible polynomials of degree 3 are $x^3 + x + 1$ and $x^3 + x^2 + 1$.
\item The irreducible polynomials of degree 4 are $x^4 + x + 1$ and $x^4 + x^3 + 1$ (note that $x^4 + x^2 + 1 = (x^2 + x + 1)^2$ is reducible).
\end{points}
\end{example}

\begin{remark}
\begin{points}
\item Every polynomial in $\Ztwox$ is a unique product of irreducible polynomials (using the Euclidean algorithm for polynomials).
\item For $k \ge 1$, $\exists$ a finite field $\field{2^k} = \Ztwoxgen{p_k(x)}$ of order $2^k$, where $p_k(x) \in \Ztwox$ has degree $k$ and is irreducible.
\item The multiplicative group $\field{2^k}^* = (\field{2^k} \setminus \set{0}, \times)$ is cyclic. If $\field{2^k}^* = \cycgen{\beta}$, then $\beta$ is called a primitive element of $\field{2^k}$.
\item Every $\gamma \in \field{2^k}$ has a minimum polynomial, which is the unique irreducible polynomial $m(x) \in \Ztwox$ satisfying $m(\gamma) = 0$. Also, $\deg m \le k$, and $m(x) \mid x^{2^k - 1} - 1$.
\end{points}
\end{remark}

\begin{example}
\begin{points}
\item Let $I = \gen{x^2 + x + 1}$, then $\field{4} = \Ztwoxover{I} = \set{0 + I, 1 + I, x + I, 1 + x + I}$. Write $\alpha = x + I$, then $\field{4} = \set{0, 1, \alpha, 1 + \alpha}$, with $\alpha^2 + \alpha + 1 = 0$.
\item $\field{8} = \Ztwoxgen{x^3 + x + 1} = \set{0, 1, \alpha, 1 + \alpha, \alpha^2, 1 + \alpha^2, \alpha + \alpha^2, 1 + \alpha + \alpha^2}$, where $\alpha = x + I$, $\alpha^3 + \alpha + 1 = 0$.
\item $\field{4}^* = \cycgen{\alpha} = \cycgen{1 + \alpha}$ has primitive elements $\alpha$ and $1 + \alpha$.
\item $\sizeof{\field{8}^*} = 7 \implies$ all its elements (except 1) are primitive.
\item Let $I = \gen{x^4 + x + 1}$, $\field{16} = \Ztwoxover{I}$, $\alpha = x + I$, then $\alpha^4 + \alpha + 1 = 0$. Also, since $\ord{\alpha} \mid \sizeof{\field{16}^*} = 15$, $\alpha^3 \ne 1$ and $\alpha^5 = \alpha^2 + \alpha \ne 1$, $\ord{\alpha} = 15 \implies \field{16}^* = \cycgen{\alpha}$.
\item In $\field{8}$, $\alpha$ and $\alpha^2$ have minimum polynomial $x^3 + x + 1$ (note that $\alpha^6 + \alpha^2 + 1 = (\alpha^3 + \alpha + 1)^2 = 0$), and $\alpha^3$ has minimum polynomial $x^3 + x^2 + 1$.
\end{points}
\end{example}

\begin{definition}
Let $k, d \in \Z_{\ge 2}$, $\beta$ be a primitive element of $\field{2^k}$, $m_i(x)$ be the minimum polynomial of $\beta^i$, $p(x) = \lcm{m_1(x), \cdots, m_{d - 1}(x)}$ and $n = 2^k - 1$, then $p(x) \mid x^n - 1$, and the cyclic code of length $n$ generated by $p(x)$ is called the BCH code of length $n$ and designed distance $d$.
\end{definition}

\begin{example}
\begin{points}
\item Let $k = 3, d = 3$. In $\field{8}$, pick a primitive element $\alpha$, then $m_1(x) = m_2(x) = x^3 + x + 1 \implies$ the BCH code is $\Ham{3}$.
\item Let $d = 4$, then $m_3(x) = x^3 + x^2 + 1 \implies p(x) = (x^3 + x + 1)(x^3 + x^2 + 1) = x^6 + \cdots + 1 \implies$ the BCH code is $\set{0^7, 1^7}$.
\end{points}
\end{example}

\begin{result}{Theorem 1.32}
Let $n = 2^k - 1$, $C$ be the BCH code of length $n$ and designed distance $d$. Then:
\noparskip
\begin{points}
\item $d(C) \ge d$,
\item $\dim C \ge n - \floor{\frac{d}{2}}k$.
\end{points}
\end{result}

\begin{proof}{Theorem 1.32}
Too hard.
\end{proof}

\begin{remark}
$\deg p \le (d - 1)k \implies \dim C = n - \deg p \ge n - (d - 1)k \implies$ the bound in Theorem 1.32 is much better than expected.
\end{remark}

\begin{example}
\begin{points}
\item Let $k = 4$, then $\exists$ a primitive element $\alpha \in \field{16}$ with minimum polynomial $x^4 + x + 1 \implies m_1(x) = m_2(x) = m_4(x) = x^4 + x + 1$, and $m_3(x) \mid x^5 - 1$ since $\ord{\alpha^3} = 5 \implies m_3(x) = x^4 + x^3 + x^2 + x + 1$.
\item Let $d = 3$, then $p(x) = \lcm{m_1, m_2} = x^4 + x + 1$ from (1) $\implies$ the BCH code $C$ has dimension $15 - \deg p = 11$ and $d(C) \ge d = 3$.
\item Let $d = 5$, then $p(x) = \lcm{m_1, m_2, m_3, m_4} = (x^4 + x + 1)(x^4 + x^3 + x^2 + x + 1)$ from (1) $\implies$ the BCH code $C$ has dimension $15 - \deg p = 7$ and $d(C) \ge d = 5$.
\end{points}
\end{example}

\begin{remark}
Since $1 + \binom{14}{1} + \binom{14}{2} + \binom{14}{3} = 1 + 14 + 91 + 364 = 470 \ge 2^{15 - 7}$, the GV bound cannot prove that $\exists$ a linear code of length 15 and dimension 7 which corrects 2 errors $\implies$ BCH beats GV.
\end{remark}

\pagebreak

\section{Strongly Regular Graphs}

\begin{definition}
\begin{points}
\item A graph $\Gamma = (V, E)$ is a set of vertices $V$ and a set of edges $E$.
\item $\Gamma$ is regular with valency $k$ if every vertex has $k$ neighbours.
\item A path in $\Gamma$ of length $r$ is a sequence of vertices $v_0, \cdots, v_r$ where $v_i$ is joined to $v_{i + 1}\ \forall i$.
\item $\Gamma$ is connected if $\exists$ a path from $v$ to $w\ \forall v, w \in V$.
\item If $\Gamma$ is connected, the distance between $v$ and $w$ is $d(v, w)$ = length of shortest path from $v$ to $w$, and the diameter of $\Gamma$ is $\diam{\Gamma} = \max \set{d(v, w): v, w \in V}$.
\item 2 graphs $(V, E)$ and $(V', E')$ are isomorphic if $\exists$ a bijection $V \rightarrow V'$ which sends $E$ to $E'$.
\end{points}
\end{definition}

\begin{example}
\begin{points}
\item \pic[0.15]{6.png} is a disconnected graph.
\item \pic[0.15]{2.png} is connected, with regular valency 2 and diameter 3.
\item \pic[0.15]{4.png} is the Petersen graph, connected with regular valency 3 and diameter 2.
\item $\diam{\Gamma} = 1 \implies$ any 2 vertices are joined by an edge. Such a graph with $v$ vertices is called the complete graph $K_v$.
\item \pic[0.15]{7.png} $\cong$ \pic[0.15]{8.png}.
\end{points}
\end{example}

\begin{result}{Proposition 2.1}
Suppose $\Gamma$ is a connected graph that is regular with valency $k$ and diameter $d$. Then $\sizeof{V(\Gamma)} \le N(k, d) = 1 + \sum_{i = 1}^d k(k - 1)^{i - 1}$.
\end{result}

\begin{proof}{Proposition 2.1}
Pick $x \in V(\Gamma)$. For $i \ge 1$, let $D_i = \set{y \in V(\Gamma): d(x, y) = i}$, then $\sizeof{D_1} = k \implies \sizeof{D_2} \le(k - 1) \sizeof{D_1} = k(k - 1) \implies \sizeof{D_3} \le(k - 1) \sizeof{D_2} = k(k - 1)^2 \implies \cdots \implies$ since $\diam{\Gamma} = d$, $V(\Gamma) = \set{x} \cup D_1 \cup \cdots \cup D_d \implies \sizeof{V(\Gamma)} = 1 + \sum_{i = 1}^d \sizeof{D_i} \le 1 + \sum_{i = 1}^d k(k - 1)^{i - 1}$.
\end{proof}

\begin{definition}
Call $\Gamma$ a Moore graph if $\Gamma$ is connected with regular valency $k$ and diameter $d$, with $\sizeof{V(\Gamma)} = N(k, d)$.
\end{definition}

\begin{example}
\begin{points}
\item Let $k = 2$, then $\sizeof{V(\Gamma)} = 1 + \sum_{i = 1}^d 2 = 2d + 1$. Indeed,\pic[0.3]{9.png} is a Moore graph.
\item Let $k = 3$, $d = 2$, then $\sizeof{V(\Gamma)} = 1 + 3 + 6 = 10$. The Petersen graph is such a graph, and is the only such Moore graph up to isomorphism.
\item Let $d = 2$, then $\sizeof{V(\Gamma)} = 1 + k + k(k - 1) = k^2 + 1$, and there are no $\triangle$s nor $\square$s in $\Gamma$. Let $k = 4$, and pick 2 joined vertices $v, w \in V(\Gamma)$, with neighbours $a, b, c$ and $x, y, z$ respectively. Since $\diam{\Gamma} = 2$, $a$ and $x$ must have a common neighbour $(a, x)$ which is a new vertex. Similarly, there are new vertices $(a, y), \cdots, (c, z)$. Also, there are 2 neighbours of $(a, x)$ among the 9 new vertices that are not of the form $(a, ?)$ nor $(?, x)$ (else there will be a $\triangle$), so the possibilities are $(b, y), (b, z), (c, y), (c, z)$. WLOG, if $(a, x)$ and $(b, y)$ are joined, then $(a, x)$ cannot be joined to $(b, z)$ nor $(c, y)$ (else there will be a $\square$) $\implies (a, x)$ and $(c, z)$ are joined. Similarly, $(b, y)$ is joined to $(a, x)$ and $(c, z)$, and $(c, z)$ is joined to $(a, x)$ and $(b, y)$ ($\contradiction$ since there is a $\triangle$) $\implies \nexists \Gamma$.
\end{points}
\end{example}

\begin{definition}
A graph $\Gamma$ is strongly regular with parameters $(v, k, a, b)$ if:
\noparskip
\begin{points}
\item $\Gamma$ has $v$ vertices,
\item $\Gamma$ is regular with valency $k$,
\item any 2 joined vertices of $\Gamma$ have $a$ common neighbours,
\item any 2 non-joined vertices of $\Gamma$ have $b$ common neighbours.
\end{points}
\end{definition}

\begin{result}{Proposition 2.2}
If $\Gamma$ is strongly regular with parameters $(v, k, a, b)$, then:
\noparskip
\begin{points}
\item $\Gamma$ is connected and $\diam{\Gamma} = 2$ if $b > 0$,
\item $\Gamma$ is a disjoint union of complete graphs $K_{k + 1}$ if $b = 0$.
\end{points}
\end{result}

\begin{proof}{Proposition 2.2}
\begin{points}
\item If $b > 0$, then $\exists$ a path of length 2 between any 2 non-joined vertices of $\Gamma \implies \diam{\Gamma} = 2$.
\item If $b = 0$, let the neighbours of a vertex $x \in V(\Gamma)$ be $x_1, \cdots, x_k$, then $x_i, x_j$ are joined $\forall i \ne j$ (else $b > 0$) $\implies x, x_1, \cdots, x_k$ form a complete graph $K_{k + 1}$. Any other vertex $y \in V(\Gamma)$ is not joined to $x \implies y$ is not joined to $x_1, \cdots, x_k$ too $\implies y$ and its neighbours for another $K_{k + 1}$.
\end{points}
\end{proof}

\begin{example}
\begin{points}
\item Moore graphs of diameter 2 are strongly regular, with parameters $(k^2 + 1, k, 0, 1)$, since there are no $\triangle$s and $\square$s.
\item For $n \ge 4$, let the $\binom{n}{2}$ pairs from $\set{1, \cdots, n}$ be vertices of $\Gamma$, and join $\set{i, j}$, $\set{k, l}$ iff $\sizeof{\set{i, j} \cap \set{k, l}} = 1$. Then $\Gamma$ is strongly regular with parameters $v = \binom{n}{2}$, $k = 2n - 4$, $a = n - 2$, $b = 4$, called the triangular graph $T(n)$.
\item Let the ordered pairs $(i, j)$ (where $i, j \in \set{1, \cdots, n}$) be vertices of $\Gamma$, and join $(i, j)$, $(k, l)$ iff $i = k$ or $j = l$. Then $\Gamma$ is strongly regular with parameters $v = n^2$, $k = 2n - 2$, $a = n - 2$, $b = 2$, called the lattice graph $L(n)$.
\item Let $p > 2$, $p \equiv 1 \pmod{4}$ be a prime, such that $\Z_p = \set{0, \cdots, p - 1}$ (with addition and multiplication mod $p$) is a field. Let $Q = \set{x^2: x \in \Z_p^*}$, $\psi: \Z_p^* \rightarrow Q$, $x \mapsto x^2$, then $\psi$ is a homomorphism with $\ker \psi = \set{x: x^2 = 1} = \set{x: (x + 1)(x - 1) = 0} = \set{\pm 1} \implies \sizeof{Q} = \sizeof{\img \psi} = \frac{\sizeof{\Z_p^*}}{\sizeof{\ker \psi}} = \frac{p - 1}{2} \equiv 0 \pmod{2} \implies -1 \in Q$, since $Q$ must contain an element of order 2. Let $V(\Gamma) = \Z_p$, and join $x, y$ iff $x - y \in Q$ (iff $y - x \in Q$), then $\Gamma$ is called the Payley graph $P(p)$.
\item $P(5)$ is \pic[0.3]{10.png}.
\end{points}
\end{example}

\begin{result}{Proposition 2.3}
$P(p)$ is strongly regular, with parameters $v = p$, $k = \frac{p - 1}{2}$, $a = \frac{p - 5}{4}$, $b = \frac{p - 1}{4}$.
\end{result}

\begin{proof}{Proposition 2.3}
Clearly $k = \sizeof{Q} = \frac{p - 1}{2}$. Now pick $x, y \in V(P(p))$ where $x \ne y$, and we aim to find the number of $z \in V(P(p))$ such that $(x, z), (y, z) \in E(P(p))$ ie. $z - x = n^2 \pmod{p}$, $z - y = m^2 \pmod{p} \implies x - y = m^2 - n^2 = (m + n)(m - n) \pmod{p}$. Since $\Z_p$ is a field, number of distinct solutions $(m + n, m - n) \in \Z_p^2$ = (number of distinct divisors of $q$) = $p - 1 \implies$ number of distinct solutions $(m, n) \in \Z_p^2 = p - 1$. But $x - y \in Q \implies (\pm m, 0), (0, \pm n)$ should be excluded (else $z = x$ or $z = y$). Also, note that $(c, d), (m, n)$ give the same value of $z \iff c^2 - m^2 \equiv (c + m)(c - m) \equiv d^2 - n^2 \equiv (d + n)(d - n) \equiv 0 \pmod{p} \iff c = \pm m, d = \pm n$. Hence $x, y$ have $\frac{(p - 1) - 4}{2^2} = \frac{p - 5}{4}$ common neighbours $z$ if they are joined, $\frac{p - 1}{2^2} = \frac{p - 1}{4}$ common neighbours otherwise $\implies P(p)$ is strongly regular, with $a = \frac{p - 5}{4}$, $b = \frac{p - 1}{4}$.
\end{proof}

\begin{result}{Proposition 2.4}
If $\Gamma$ is strongly regular with parameters $(v, k, a, b)$, then $k(k - a - 1) = b(v - k - 1)$.
\end{result}

\begin{proof}{Proposition 2.4}
Pick $x \in V(\Gamma)$, and let $A$ be the set of $k$ neighbours of $x$, $B$ be the set of $v - k - 1$ non-neighbours of $x$, $N$ be the number of edges joining a vertex in $A$ to a vertex in $B$. Each vertex in $A$ is joined to $k - a - 1$ vertices in $B$, and each vertex in $B$ is joined to $b$ vertices in $A \implies (k - a - 1) \sizeof{A} = N = b \sizeof{B} \implies k(k - a - 1) = b(v - k - 1)$.
\end{proof}

\begin{example}
Moore graphs of diameter 2 are strongly regular, with parameters $(v, k, 0, 1) \implies k(k - 1) = v - k - 1 \implies v = k^2 + 1$ indeed.
\end{example}

\begin{remark}
We can draw the ``balloon" picture \pic[0.3]{11.png} if $\Gamma$ is strongly regular.
\end{remark}

\begin{definition}
Replace all edges of $\Gamma$ with non-edges and vice-versa but keep the same vertex set, then the new graph obtained is $\Gamma^c$, called the complement of $\Gamma$.
\end{definition}

\begin{result}{Proposition 2.5}
If $\Gamma$ is strongly regular with parameters $(v, k, a, b)$, then $\Gamma^c$ is also strongly regular, with parameters $(v, v - k - 1, v - 2k + b - 2, v - 2k + a)$.
\end{result}

\begin{proof}{Proposition 2.5}
Pick $x \in \Gamma^c$, and let $B$ be the set of neighbours of $x$ in $\Gamma^c$ (ie. the set of non-neighbours of $x$ in $\Gamma$), $A$ be the set of non-neighbours of $x$ in $\Gamma^c$ (ie. the set of neighbours of $x$ in $\Gamma$), then clearly $\sizeof{B} = v - k - 1$. Also, in $\Gamma$, any vertex $v \in A$ is joined to $k - a - 1$ vertices in $B \implies$ in $\Gamma^c$, $v$ is joined to $\sizeof{B} - (k - a - 1) = v - k - 1 - k + a + 1 = v - 2k + a$ vertices in $B$. Moreover, in $\Gamma$, any vertex $w \in B$ is joined to $k - b$ other vertices in $B \implies$ in $\Gamma^c$, $w$ is joined to $\sizeof{B} - (k - b) - 1 = v - k - 1 - k + b - 1 = v - 2k + b - 2$ vertices in $B$. Hence $\Gamma^c$ is strongly regular, with parameters $(v, v - k - 1, v - 2k + b - 2, v - 2k + a)$.
\end{proof}

\subsection{Adjacency matrices}

\begin{definition}
Let $\Gamma$ be a graph with vertex set $\set{e_1, \cdots, e_v}$. The adjacency matrix of $\Gamma$ is the $v \times v$ matrix $A = (a_{ij})$, with $a_{ij} = 1$ if $e_i$ is joined to $e_j$, 0 otherwise.
\end{definition}

\begin{example}
The adjacency matrix for \pic[0.3]{12.png} is $A = \begin{pmatrix}
0 & 1 & 0 & 0 & 1 \\
1 & 0 & 1 & 0 & 0 \\
0 & 1 & 0 & 1 & 0 \\
0 & 0 & 1 & 0 & 1 \\
1 & 0 & 0 & 1 & 0 \\
\end{pmatrix}$.
\end{example}

\begin{remark}
\begin{points}
\item $A$ is symmetric, with all entries 0 or 1.
\item $A$ has 0's on its main diagonal.
\end{points}
\end{remark}

\begin{result}{Proposition 2.6}
Let $\Gamma$ be a strongly regular graph with parameters $(v, k, a, b)$, $A$ be its adjacency matrix, and $J$ be the $v \times v$ matrix consisting of all 1's. Then:
\noparskip
\begin{points}
\item $AJ = kJ$,
\item $A^2 = (a - b)A + (k - b)I + bJ$.
\end{points}
\end{result}

\begin{proof}{Proposition 2.6}
\begin{points}
\item $\Gamma$ is regular with valency $k \implies$ each row of $A$ has exactly $k$ 1's $\implies AJ = kJ$.
\item Since $A$ is symmetric, $(A^2)_{ij} = (A A^\top)_{ij}$ = (row $i$ of $A$) $\cdot$ (column $j$ of $A^\top$) = (row $i$ of $A$) $\cdot$ (row $j$ of $A$) = $k$ if $i = j$, $a$ if $i \ne j$ and $e_i, e_j$ are joined, $b$ otherwise $\implies A^2$ has $k$'s on its main diagonal, $a$'s where $A$ has 1's, and $b$'s where $A$ has 0's $\implies A^2 = kI + aA + b(J - A - I) = (a - b)A + (k - b)I + bJ$.
\end{points}
\end{proof}

\subsection{Eigenvalues of adjacency matrices}

The adjacency matrix $A$ of a graph $\Gamma$ is real and symmetric, so it has real eigenvalues and is diagonalizable.

\begin{definition}
The multiplicity of an eigenvalue $\lambda$ is the number of times it appears in $\begin{pmatrix}
\lambda_1 &           &             0 \\
                & \ddots &                \\
             0 &           & \lambda_v \\
\end{pmatrix} = P^{-1}AP$ for some $P$.
\end{definition}

\begin{result}{Lemma 2.7}
Let $A$ be a real $v \times v$ matrix with eigenvalues $\lambda_1, \cdots, \lambda_v$, then $\trace{A} = \sum_{i = 1}^v \lambda_i$.
\end{result}

\begin{proof}{Lemma 2.7}
$\lambda_1, \cdots, \lambda_v$ are the roots of $\det (xI - A) = (x - a_{11}) \cdots (x - a_{vv}) + ($terms of degree $\le v - 2) = x^v - (a_{11} + \cdots + a_{vv}) x^{v - 1} + \cdots$. Since $\det (xI - A)$ is also = $(x - \lambda_1) \cdots (x - \lambda_v)$, comparing coefficients of $x^{v - 1}$ gives $\sum_{i = 1}^v \lambda_i = \sum_{i = 1}^v a_{ii} = \trace{A}$.
\end{proof}

\begin{result}{Theorem 2.8}
Let $\Gamma$ be a strongly regular graph with parameters $(v, k, a, b)$ and adjacency matrix $A$. Assume WLOG that $v > 2k$ (else pick $\Gamma^c$), and suppose $\Gamma$ is connected (ie. $b > 0$). Then:
\noparskip
\begin{points}
\item $A$ has exactly 3 distinct eigenvalues $k, r_1, r_2$, where $r_1, r_2$ satisfy $x^2 - (a - b)x - (k - b) = 0$,
\item eigenvalue $k$ has multiplicity 1, and if $m_1, m_2$ are the multiplicities of $r_1, r_2$ respectively, then $m_1 + m_2 = v - 1$ and $m_1 r_1 + m_2 r_2 = -k$,
\item $r_1, r_2 \in \Z$ unless $(v, k, a, b)$ has the form $(4b + 1, 2b, b - 1, b)$.
\end{points}
\end{result}

\begin{proof}{Theorem 2.8}
By Proposition 2.6(1), $AJ = kJ \implies$ let $\vec{j} = \begin{pmatrix} 1 & \cdots & 1 \end{pmatrix}^\top$, then $A\vec{j} = k\vec{j} \implies k$ is an eigenvalue of $A$. \\
Now let $\vec{w}$ be an eigenvector of $A$ with $\vec{w} \notin \span{\vec{j}}$ such that $A\vec{w} = \lambda \vec{w}$, then by Proposition 2.6(2), $A^2 \vec{w} = (a - b)A\vec{w} + (k - b)I\vec{w} + bJ\vec{w} \implies \lambda^2 \vec{w} = (a - b) \lambda \vec{w} + (k - b) \vec{w} + b(c\vec{j})$ (where $c$ = sum of coordinates of $\vec{w}$) $\implies (\lambda^2 - (a - b) \lambda - (k - b)) \vec{w} = bc\vec{j} \in \span{\vec{j}} \implies$ since $\vec{w} \notin \span{\vec{j}}$, $\lambda^2 - (a - b) \lambda - (k - b) = 0$ ie. $\lambda$ satisfies $x^2 - (a - b)x - (k - b) = 0$. \\
Let the roots of $x^2 - (a - b)x - (k - b) = 0$ be $r_1$ and $r_2$, and suppose $k = r_1$ or $r_2 \implies k^2 - (a - b)k - (k - b) = 0$. But by Proposition 2.4, $k(k - a - 1) = b(v - k - 1) \implies k^2 - (a - b)k - (k - b) = bv \implies bv = 0 \implies b = 0$ ($\contradiction$) $\implies k \ne r_1, r_2 \implies$ the eigenspace for $k$ is $\span{\vec{j}} \implies k$ has multiplicity 1. Moreover, if $r_1 = r_2$, then $(a - b)^2 + 4(k - b) = 0 \implies$ since $k \ge b$, we must have $(a - b)^2 = 4(k - b) = 0 \implies a = b = k$ ($\contradiction$ as $k - 1 \ge a$) $\implies r_1 \ne r_2$. Hence $k ,r_1, r_2$ are all distinct. \\
Let the multiplicities of $r_1$ and $r_2$ be $m_1$ and $m_2$ respectively, then $m_1 + m_2 = v - 1$ since $A$ is a $v \times v$ matrix, and $m_1 r_1 + m_2 r_2 + k = \trace{A} = 0 \implies m_1 r_1 + m_2 r_2 = -k$. \\
Now let $D = (a - b)^2 + 4(k - b)$, then $r_1, r_2 = \frac{1}{2} ((a - b) \pm \sqrt{D}) \implies 2(m_1 r_1 + m_2 r_2) = (m_1 + m_2)(a - b) + (m_1 - m_2) \sqrt{D} = -2k \implies$ if $m_1 \ne m_2$, then $\sqrt{D} \in \Q \implies \sqrt{D} \in \Z \implies r_1, r_2$ are either both $\in \Z$ or both of the form $z + \frac{1}{2}$ for some $z \in \Z$. If the latter is true, then $r_1 r_2 = -(k - b) \notin \Z$ ($\contradiction$) $\implies r_1, r_2 \in \Z$. In particular, if $m_2 = 0$, then $m_1 = v - 1$ and $m_1 r_1 = -k \implies v - 1 \mid k \implies v - 1 \le k$ ($\contradiction$ since $v > 2k$) $\implies m_1, m_2 > 0 \implies A$ has exactly 3 eigenvalues. \\
Otherwise, if $m_1 = m_2 = m$, then $2m = v - 1$ and $2m(a - b) = -2k \implies (v - 1)(a - b) = -2k \implies v - 1 \le 2k$. Since $v > 2k$ by assumption, we must have $v = 2k + 1 \implies a - b = -1 \implies$ by Proposition 2.4, $k(k - a - 1) = b(v - k - 1) = bk \implies b = k - a - 1 = k - b \implies k = 2b \implies (v, k, a, b) = (4b + 1, 2b, b - 1, b)$.
\end{proof}

\begin{remark}
If $v \le 2k$, then $\Gamma^c$ is strongly regular, and $v' = v > v + (v - 2k - 2) = 2(v - k - 1) = 2k' \implies$ Theorem 2.8 applies to $\Gamma^c$ if it is connected. Otherwise, $\Gamma^c$ is a disjoint union of complete graphs $\implies$ we know what $\Gamma^c$ is.
\end{remark}

\begin{result}{Theorem 2.9}
If $\exists$ a Moore graph of valency $k$ and diameter 2, then $k$ = 2, 3, 7 or 57.
\end{result}

\begin{proof}{Theorem 2.9}
Let $\Gamma$ be such a Moore graph, then $\Gamma$ is strongly regular with parameters $(k^2 + 1, k, 0, 1)$. Let $A$ be the adjacency matrix of $\Gamma$, then since $b > 0$ and $k^2 + 1 > 2k$ for $k > 1$, by Theorem 2.8(1), $A$ has 3 eigenvalues $k, r_1, r_2$ where $r_1, r_2$ are the roots of $x^2 + x - (k - 1) = 0 \implies r_1, r_2 = \frac{1}{2} (-1 \pm \sqrt{4k - 3})$. Also, by Theorem 2.8(2), the multiplicities of $r_1, r_2$ satisfy $m_1 + m_2 = k^2$ and $m_1 r_1 + m_2 r_2 = -k \implies \frac{1}{2} (-m_1 - m_2) + \frac{1}{2} (m_1 - m_2) \sqrt{4k - 3} = -k \implies (m_1 - m_2) \sqrt{4k - 3} = k^2 - 2k$. \\
If $k = 2b = 2$, then $\Gamma$ = \pic[0.15]{8.png}. Otherwise, if $k > 2$, $r_1, r_2 \in \Z \implies \sqrt{4k - 3} \in \Z$ by Theorem 2.8(3). Let $n = \sqrt{4k - 3}$, then $k = \frac{n^2 + 3}{4} \implies n(m_1 - m_2) = k(k - 2) = \frac{n^2 + 3}{4} \times \frac{n^2 - 5}{4} \implies m_1 - m_2 = \frac{(n^2 + 3)(n^2 - 5)}{16n} \in \Z \implies n \mid (n^2 + 3)(n^2 - 5) \implies n \mid 15 \implies n$ = 1, 3, 5 or 15 $\implies k$ = 1 ($\contradiction$), 3, 7 or 57. Hence $k$ = 2, 3, 7 or 57.
\end{proof}

\begin{result}{Theorem 2.10 (Friendship Theorem)}
If $\Gamma$ is a graph in which any 2 vertices have exactly 1 common neighbour, then $\exists$ a vertex that is joined to all the other vertices.
\end{result}

\begin{proof}{Theorem 2.10}
Such $\exists$ such a $\Gamma$ but no vertex is joined to all the other vertices. Let $v(P)$ be the number of neighbours of $P$, and $R$ be the common neighbour of $P$ and $Q$, where $P$ and $Q$ are not joined. Let $S$ be the common neighbour of $P$ and $R$, and $T$ be the common neighbour of $Q$ and $R$, then $S \ne Q$, $T \ne P$ and $S \ne T$ (else $P$ and $Q$ have 2 common neighbours) ie. we have \pic[0.3]{13.png}. Let the remaining neighbours of $P$ be $u_1, \cdots, u_r$, then the common neighbour of $u_1$ and $Q$ cannot be $T$ (else $P$ and $T$ have 2 common neighbours) nor $R$ (else $P$ and $R$ have 2 common neighbours) $\implies$ it is a new vertex $v_1$. Similarly, the common neighbour of $u_2$ and $Q$ cannot be $T, R$ nor $v_1$ (else $P$ and $v_1$ have 2 common neighbours) $\implies$ it is a new vertex $v_2 \implies$ repeating $\forall u_i$ gives $v(P) = r + 2 \le v(Q)$. Likewise, $v(Q) \le v(P) \implies v(P) = v(Q) \implies$ any 2 non-joined vertices have the same number of neighbours. \\
Now let $B$ be a vertex that is not $P$, $Q$ nor $R$, then $v(B) = v(P) = v(Q)$ since $B$ is not joined to either $P$ or $Q$ (or both). Also, by assumption, let $C$ be a vertex that is not joined to $R$, then $v(Q) = v(C) = v(R)$ too $\implies$ every vertex has the same number of neighbours as $Q \implies \Gamma$ is regular $\implies \Gamma$ is strongly regular with parameters $(v, k, 1, 1)$. \\
By Proposition 2.4, $k(k - 2) = v - k - 1 \implies v = k^2 - k + 1 \implies v > 2k$ iff $k \ge 3$. If $k = 2$, then $\Gamma = \triangle$ ($\contradiction$) $\implies v > 2k \implies$ let $A$ be the adjacency matrix of $\Gamma$, then by Theorem 2.8(1), $A$ has 3 eigenvalues $k, r_1, r_2$ where $r_1, r_2$ are the roots of $x^2 - (k - 1) = 0 \implies r_1 = \sqrt{k - 1}$, $r_2 = -\sqrt{k - 1}$. Also, by Theorem 2.8(2), $m_1 + m_2 = v - 1 = k^2 - k$ and $m_1 r_1 + m_2 r_2 = -k \implies (m_1 - m_2) \sqrt{k - 1} = -k \implies (m_1 - m_2)^2 (k - 1) = k^2 \implies k - 1 \mid k^2 \implies k - 1 \mid 1 \implies k$ = 0 or 2 ($\contradiction$) $\implies \nexists \Gamma$.
\end{proof}

\subsection{Strongly regular graphs with small $v$}

\begin{example}
\begin{points}
\item $T(6)$ has parameters $(15, 8, 4, 4)$.
\item $T(6)^c$ has parameters $(15, 6, 1, 3)$.
\item $(K_3)^5$ has parameters $(15, 2, 1, 0)$, and $(K_5)^3$ has parameters $(15, 4, 3, 0)$.
\item $\left[ (K_3)^5 \right]^c$ has parameters $(15, 12, 9, 12)$, and $\left[ (K_5)^3 \right]^c$ has parameters $(15, 10, 5, 10)$.
\end{points}
\end{example}

\begin{result}{Proposition 2.11}
If $\Gamma$ is strongly regular with $v = 15$, then $\Gamma = T(6), (K_3)^5, (K_5)^3$ or their complements.
\end{result}

\begin{proof}{Proposition 2.11}
Let $\Gamma$ have parameters $(15, k, a, b)$. If $15 \le 2k$, replace $\Gamma$ by $\Gamma^c \implies$ assume WLOG that $15 > 2k$. If $b = 0$, then $\Gamma = (K_3)^5$ or $(K_5)^3$ by Proposition 2.2. If $b > 0$, then $2 \le k \le 7$. \\
If $k = 2$, then $\Gamma$ is a 15-gon ($\contradiction$ since $\Gamma$ is not strongly regular). \\
If $k = 3$, by Proposition 2.4, $3(2 - a) = 11b \implies 11 \mid 2 - a \implies a = 2 \implies b = 0$ ($\contradiction$). \\
If $k = 4$, by Proposition 2.4, $4(3 - a) = 10b \implies 5 \mid 3 - a \implies a = 3 \implies b = 0$ ($\contradiction$). \\
If $k = 5$, by Proposition 2.4, $5(4 - a) = 9b \implies 9 \mid 4 - a \implies a = 4 \implies b = 0$ ($\contradiction$). \\
If $k = 6$, by Proposition 2.4, $6(5 - a) = 8b \implies 8 \mid 5 - a \implies a = 1 \implies b = 3 \implies \Gamma = T(6)^c$. \\
If $k = 7$, by Proposition 2.4, $7(6 - a) = 7b \implies b = 6 - a$. Also, by Theorem 2.8, the eigenvalues of the adjacency matrix of $\Gamma$ are $7, r_1, r_2$ where $r_1, r_2$ are the roots of $x^2 - (a - b)x - (k - b) = x^2 - (2a - 6)x - (1 + a) = 0 \implies r_1, r_2 = a - 3 \pm \sqrt{(a - 3)^2 + (1 + a)} = a - 3 \pm \sqrt{a^2 - 5a + 10}$. Since $r_1, r_2 \in \Z$, $a^2 - 5a + 10$ is a perfect square, and $0 \le a \le k - 1 = 5 \implies a$ = 2 or 3. If $a = 2$, then $r_1 = 1$, $r_2 = -3 \implies m_1 + m_2 = 14$ and $m_1 - 3m_2 = -7 \implies 4m_2 = 21$ ($\contradiction$). If $a = 3$, then $r_1 = 2$, $r_2 = -2 \implies m_1 + m_2 = 14$ and $2m_1 - 2m_2 = -7 \implies 4m_1 = 21$ ($\contradiction$). \\
Hence $\Gamma = T(6)^c, (K_3)^5$ or $(K_5)^3 \implies \Gamma = T(6), (K_3)^5, (K_5)^3$ or their complements.
\end{proof}

\subsection{2-weight codes \& strongly regular graphs}

\begin{definition}
A linear code $C \subseteq \Z_2^n$ is a 2-weight code if $\exists w_1, w_2 > 0$, $w_1 \ne w_2$, such that $\wt{c} = w_1$ or $w_2\ \forall c \in C \setminus \set{0}$, and both occur.
\end{definition}

\begin{example}
\begin{points}
\item $H' \subseteq \Z_2^8$ has weights 0, 4 or 8 $\implies$ it is a 2-weight code.
\item $C = \set{v \in \Z_2^5: \wt{v} \text{ even}}$ is a 2-weight code.
\item $C = \set{c \in G_{24}: c_{16} = \cdots = c_{24} = 0}$ has weights 0, 8 or 12 $\implies C$ is a 2-weight code.
\end{points}
\end{example}

\begin{definition}
A linear code is projective if it has a generator matrix whose columns are distinct and non-zero.
\end{definition}

\begin{example}
$H'$ is projective, with generator matrix $\begin{pmatrix}
1 & 1 & 1 & 1 & 1 & 1 & 1 & 1 \\
0 & 1 & 1 & 1 & 0 & 0 & 0 & 1 \\
0 & 1 & 0 & 0 & 1 & 1 & 0 & 1 \\
0 & 0 & 1 & 0 & 1 & 0 & 1 & 1 \\
\end{pmatrix}$.
\end{example}

\begin{result}{Theorem 2.12}
Let $C \subseteq \Z_2^n$ be a projective 2-weight linear code, with weights $w_1, w_2 > 0$, $w_1 \ne w_2$. Define $\Gamma$ with $V(\Gamma) = C$, and join $a, b$ iff $d(a, b) = \wt{a + b} = w_1$, then $\Gamma$ is strongly regular.
\end{result}

\begin{proof}{Theorem 2.12}
Let $\dim C = k$ and $b_i$ be the number of codewords of weight $w_i$ for $i = 1, 2$, then clearly $\Gamma$ is regular with valency $b_1$. Define $A_i$ to be the $b_i \times n$ matrix whose rows are the codewords with weight $w_i$, $A = \begin{pmatrix} A_1 \\ A_2 \end{pmatrix}$, and $\phi_t: C \rightarrow \Z_2$ by $\phi_t(x_1, \cdots, x_n) = x_t$. Since every column of $A$ has a non-zero element, $\phi_t$ is surjective $\implies \dim \ker \phi_t = k - 1 \implies$ each column of $A$ has $2^{k - 1} - 1$ 0's and $2^{k - 1}$ 1's $\implies$ number of 1's in $A = n \times 2^{k - 1} = b_1 w_1 + b_2 w_2 \implies$ we can solve for $b_i$ as $b_1 + b_2 = 2^k - 1$. \\
For any fixed $j$, let $r_i$ be the number of 0's in column $j$ of $A_i$. Since codewords $c \in C$ with $c_j = 0$ form a 2-weight projective subcode of $C$, we can calculate $r_i$ in the same way we did for $b_i$ (ie. $r_1 + r_2 = 2^{k - 1} - 1$ and $r_1 w_1 + r_2 w_2 = (n - 1) \times 2^{k - 2}$) $\implies$ every column of $A_i$ has $r_i$ 0's. Let $A_1$ have rows $a_1, \cdots, a_{b_1}$, $s_p$ = number of $a_m$ such that $d(a_j, a_m) = w_p$, then $s_1 + s_2 = b_1 - 1$ and $s_1 w_1 + s_2 w_2 = \sum_{m = 1}^{b_1} d(a_j, a_m) = w_1 r_1 + (n - w_1)(b_1 - r_1) \implies$ we can solve for $s_p$. \\
It follows that $a_j$ and $0^n$ are joined and have $s_1$ common neighbours $\forall j$. Moreover, for any edge $(x, y)$ with $\wt{x + y} = w_1$, $z$ is a common neighbour of $x$ and $y$ iff $x + z$ is a common neighbour of $0^n$ and $x + y \implies$ any pair of joined vertices have $s_1$ common neighbours. Likewise for $A_2$, any pair of non-joined vertices has a constant number of common neighbours $\implies \Gamma$ is strongly regular.
\end{proof}

\begin{example}
Let $C = H'$, $w_1 = 8$, $w_2 = 4$, and join $a, b \in \Gamma^c$ iff $d(a, b) = 8$ ie. $a = b + 1^8$, then $\Gamma^c$ has valency 1, and in fact $\Gamma^c = (K_2)^8 \implies \Gamma$ is also strongly regular.
\end{example}

\pagebreak

\section{Designs}

\begin{definition}
Let $X$ be a set of $v$ points, then a $t$-design with parameters $(v, k, r_t)$ is a collection $\cB$ of subsets of $X$, all of which have size $k$ (called blocks), such that any $t$ points of $X$ lie in $r_t$ blocks.
\end{definition}

\begin{remark}
$\cB$ is trivial if every set of size $k$ is a block.
\end{remark}

\begin{example}
\begin{points}
\item Octads in $G_{24}$ form a 5-design with parameters $(24, 8, 1)$.
\item Let $X = \Z_2^n \setminus \set{0}$ with blocks $\set{x, y, x + y}$, then $\cB$ is a 2-design with parameters $(2^k - 1, 3, 1)$.
\end{points}
\end{example}

\begin{result}{Proposition 3.1}
A $t$-design is also an $s$-design $\forall 1 \le s \le t$, and $r_s = \frac{(v - t + 1) \cdots (v - s)}{(k - t + 1) \cdots (k - s)} r_t$.
\end{result}

\begin{proof}{Proposition 3.1}
Follows from Corollary 1.24.
\end{proof}

\begin{notation}
Write $r = r_1$, $b = r_0$ = number of blocks, then $bk = vr$ by Proposition 3.1.
\end{notation}

\begin{example}
\begin{points}
\item $\nexists$ a 2-design with parameters $(56, 11, 1)$ since $r = r_1 = \frac{56 - 2 + 1}{11 - 2 + 1} \times 1 = \frac{55}{10} \notin \Z$.
\item Consider a 2-design with parameters $(46, 10, 1)$, then $r = \frac{45}{9} = 5 \implies b = \frac{vr}{k} = \frac{46 \times 5}{10} = 23 \implies$ we do not know if it exists.
\end{points}
\end{example}

\subsection{Some theory of 2-designs}

\begin{notation}
Write $r_2 = \lambda$, such that the parameters of a 2-design become $(v, k, \lambda)$.
\end{notation}

\begin{result}{Proposition 3.2}
For a 2-design, $r(k - 1) = \lambda (v - 1)$.
\end{result}

\begin{proof}{Proposition 3.2}
Consider pairs $(ij, B)$ where $B \in \cB$, $i, j \in B$, $i \ne j$, then the number of such pairs is = ways to choose $i, j \in X\ \times$ ways to choose $B$ containing $i, j = \binom{v}{2} \lambda$. \\
On the other hand, this number is also = ways to choose $B \in \cB\ \times$ ways to choose $i, j \in B = b \binom{k}{2} \implies$ by Proposition 3.1, $\frac{\lambda v(v - 1)}{2} = \frac{bk(k - 1)}{2} = \frac{vr(k - 1)}{2} \implies \lambda (v - 1) = r(k - 1)$.
\end{proof}

\subsection{Incidence matrices}

\begin{definition}
Let $\cB$ be a $t$-design ($t \ge 1$) with points $x_1, \cdots, x_v$ and blocks $B_1, \cdots, B_b$, then the incidence matrix of $\cB$ is the $v \times b$ matrix $A = (a_{ij})$, with $a_{ij} = 1$ if $x_i \in B_j$, 0 otherwise.
\end{definition}

\begin{remark}
Each row of $A$ has $r$ 1's, and each column of $A$ has $k$ 1's.
\end{remark}

\begin{result}{Proposition 3.3}
Let $\cB$ be a 2-design with parameters $(v, k, \lambda)$ and incidence matrix $A$, then $AA^\top$ (a $v \times v$ matrix) = $\lambda J_v + (r - \lambda) I_v$.
\end{result}

\begin{proof}{Proposition 3.3}
$(AA^\top)_{ij}$ = (row $i$ of $A$) $\cdot$ (row $j$ of $A$) = (number of blocks containing both $i$ and $j$) = $\lambda$ if $i \ne j$, $r$ otherwise $\implies$ the result follows.
\end{proof}

\begin{result}{Proposition 3.4}
Let $A$ be the incidence matrix of a 2-design with parameters $(v, k, \lambda)$, then $\det{AA^\top} = (r - \lambda)^{v - 1} (r + (v - 1) \lambda)$.
\end{result}

\begin{proof}{Proposition 3.4}
$\begin{vmatrix}
          r &           & \lambda \\
            & \ddots &              \\
\lambda &          &            r \\
\end{vmatrix} = \begin{vmatrix}
           r & \lambda - r & \cdots & \lambda - r \\
\lambda & r - \lambda &           &               0 \\
   \vdots &                  & \ddots &                  \\
\lambda &               0 &           & r - \lambda \\
\end{vmatrix} = \begin{vmatrix}
r + (v - 1) \lambda &                  &           &               0 \\
               \lambda & r - \lambda &           &                  \\
                  \vdots &                  & \ddots &                  \\
               \lambda &               0 &           & r - \lambda \\
\end{vmatrix} = (r - \lambda)^{v - 1} (r + (v - 1) \lambda)$.
\end{proof}

\begin{result}{Theorem 3.5 (Fisher's Inequality)}
Let $\cB$ a 2-design with parameters $(v, k, \lambda)$, with $v > k$, then $b \ge v$ (and $r \ge k$).
\end{result}

\begin{proof}{Theorem 3.5}
By Proposition 3.2, $v > k \implies r > \lambda$. Let $A$ be the incidence matrix of $\cB$, then by Proposition 3.4, $\det{AA^\top} > 0 \implies AA^\top$ is invertible $\implies v = \rank AA^\top \le \rank A \le b$.
\end{proof}

\begin{example}
From the previous example, a 2-design with parameters $(46, 10, 1)$ must have $b = 23 < v$ ($\contradiction$) $\implies$ no such design exists.
\end{example}

\subsection{Symmetric 2-designs}

\begin{definition}
A 2-design is symmetric if $v = b$ (or equivalently $k = r$).
\end{definition}

\wrappic[0.3]{14.png}
\begin{example}
Let $X = \Z_2^3 \setminus \set{0}$ with blocks $\set{x, y, x + y}$, then $\cB$ is a 2-design with parameters $(7, 3, 1)$. In addition, $r(k - 1) = \lambda (v - 1) \implies (3 - 1)r = (7 - 1) \implies r = 3 = k \implies \cB$ is a symmetric 2-design. $\cB$ is also called the Fano plane, and is the smallest projective plane ie. a symmetric 2-design with $\lambda = 1$.
\end{example}

\begin{result}{Theorem 3.6}
If $\exists$ a symmetric 2-design with parameters $(v, k, \lambda)$ where $v$ is even, then $k - \lambda$ is a square.
\end{result}

\begin{proof}{Theorem 3.6}
Since $b = v$, the incidence matrix $A$ of such a design is $v \times v \implies \det{A}$ exists and is $\in \Z$. By Proposition 3.4 and Proposition 3.2, $\det{A}^2 = \det{A} \det{A^\top} = \det{AA^\top} = (r - \lambda)^{v - 1} (r + r(k - 1)) = (k - \lambda)^{v - 1} (k + k(k - 1)) = (k - \lambda)^{v - 1} k^2 \implies (k - \lambda)^{v - 1}$ is a square $\implies$ since $v - 1$ is odd, $k - \lambda$ must be a square.
\end{proof}

\begin{example}
Suppose $\cB$ is a 2-design with parameters $(22, 7, 2)$, then by Proposition 3.2, $r(k - 1) = \lambda (v - 1) \implies (7 - 1)r = 2(22 - 1) \implies r = 7 = k \implies \cB$ is symmetric. But $v$ is even and $k - \lambda = 5$ is not a square $\implies$ by Theorem 3.6, $\nexists \cB$.
\end{example}

\begin{remark}
If $v$ is odd, then the Bruck-Ryser-Chowla Theorem says that if a symmetric 2-design with parameters $(v, k, \lambda)$ exists, then $z^2 = (k - \lambda)x^2 + (-1)^\frac{v - 1}{2} \lambda y^2$ has a non-zero solution for $x, y, z \in \Z$.
\end{remark}

\begin{result}{Theorem 3.7}
If $\cB$ is a symmetric 2-design with parameters $(v, k, \lambda)$, then any 2 blocks of $\cB$ intersect at exactly $\lambda$ points.
\end{result}

\begin{proof}{Theorem 3.7}
Let $A$ be the $v \times v$ incidence matrix of $\cB = \set{B_1, \cdots, B_v}$, and consider $A^\top A$. Since $JA = kJ = rJ = AJ$ and $IA = AI$, $A(A^\top A) = (AA^\top)A = (\lambda J + (r - \lambda) I)A = A(\lambda J + (r - \lambda) I) = A(AA^\top)$ by Proposition 3.3. From the proof of Proposition 3.6, since $\det{A}^2 = (k - \lambda)^{v - 1} k^2 \ne 0$ (else $r = k = \lambda \implies k = v \implies \cB$ is a trivial design with $b = 1$), $A$ is invertible $\implies A^\top A = AA^\top \implies \sizeof{B_i \cap B_j}$ = (column $i$ of $A$) $\cdot$ (column $j$ of $A$) = $(A^\top A)_{ij} = (AA^\top)_{ij} = \lambda\ \forall i \ne j$.
\end{proof}

\subsection{Difference sets}

\begin{example}
Let $X = \Z_7$, $B_0 = \set{0, 1, 3} \subseteq X$. For $0 \le i \le 6$, define $B_0 + i = \set{b + i: b \in B_0}$, then these 7 subsets of $X$ form the blocks of a symmetric 2-design with parameters $(7, 3, 1)$.
\end{example}

\begin{definition}
Let $\lambda, v \in \Z^+$, $B_0 \subseteq \Z_v$. Call $B_0$ a $\lambda$-difference set if $\forall d \in \Z_v \setminus \set{0}$, there are exactly $\lambda$ pairs $(b_1, b_2) \in B_0 \times B_0$ such that $b_1 - b_2 = d$.
\end{definition}

\begin{result}{Proposition 3.8}
Suppose $B_0$ is a $\lambda$-difference set in $\Z_v$. Let $k = \sizeof{B_0}$, and for $i \in \Z_v$, define $B_0 + i = \set{b + i: b \in B_0}$. Then the subsets $B_0 + i$ form the blocks of a symmetric 2-design with parameters $(v, k, \lambda)$.
\end{result}

\begin{proof}{Proposition 3.8}
All $v$ subsets $B_0 + i$ have size $k$, so it suffices to show that any 2 points in $\Z_v$ lie in $\lambda$ blocks. Pick $r, s \in \Z_v$, $r \ne s$, then $r, s \in B_0 + i \iff r - i, s - i \in B_0 \implies$ (number of choices for $i$) = (number of pairs $\in B_0 \times B_0$ with difference $r - s$) = $\lambda \implies$ the result follows.
\end{proof}

\begin{example}
\begin{points}
\item Let $v = 11$, $B_0 = \set{1, 4, 9, 5, 3} \subseteq \Z_{11}$, then by Proposition 3.8, since $B_0$ is a 2-difference set, we have a symmetric 2-design with parameters $(11, 5, 2)$.
\item Let $v = 13$, $B_0 = \set{0, 1, 3, 9} \subseteq \Z_{13}$, then $B_0$ is a 1-difference set $\implies$ we have a symmetric 2-design with parameters $(13, 4, 1)$.
\end{points}
\end{example}

\begin{result}{Proposition 3.9}
Let $p$ be a prime, $Q = \set{x^2: x \in \Z_p \setminus \set{0}}$. If $p \equiv 3 \pmod{4}$, then $Q$ is a $\frac{p - 3}{4}$-difference set, and the corresponding symmetric 2-design has parameters $\left( p, \frac{p - 1}{2}, \frac{p - 3}{4} \right)$.
\end{result}

\begin{proof}{Proposition 3.9}
Note that $Q \le (\Z_p^*, \times)$, and $\sizeof{Q} = \frac{p - 1}{2} \equiv 1 \pmod{2} \implies -1 \notin Q \implies Q \cup (-Q) = \Z_p^*$. For $q \in Q$, define $S_q = \set{(x_1, x_2) \in Q \times Q: x_1 - x_2 = q}$. Since $r \in Q \implies qr \in Q$ and $x_1 - x_2 = q \iff rx_1 - rx_2 = rq$, we have $(x_1, x_2) \in S_q \iff (rx_1, rx_2) \in S_{rq} \implies \sizeof{S_q} = \sizeof{S_{rq}} \implies \sizeof{S_q}$ is constant for $q \in Q$.
Moreover, $-q \in -Q$, and $(x_1, x_2) \in S_q \iff (x_2, x_1) \in S_{-q} \implies \sizeof{S_q} = \sizeof{S_{-q}} \implies \sizeof{S_x}$ is constant $\forall x \in Q \cup (-Q) = \Z_p^* \implies Q$ is a difference set in $\Z_p$, with $\lambda = \frac{\sizeof{Q} \times (\sizeof{Q} - 1)}{\sizeof{\Z_p^*}} = \frac{p - 1}{2} \times \frac{p - 3}{2} \div (p - 1) = \frac{p - 3}{4} \implies$ the result follows.
\end{proof}

\subsection{Affine planes}

\begin{definition}
Let $F$ be a finite field, then $F^2 = \set{(x_1, x_2): x_1, x_2 \in F}$ is a 2-dimensional vector space over $F$. Define points to be vectors in $F^2$ and lines to be subsets of the form $\set{v + \lambda w: \lambda \in F} \subseteq F^2$ for some fixed $v, w \in F^2$, then this collection of points and lines is called the affine plane over $F$, denoted $AG(2, F)$.
\end{definition}

\begin{remark}
\begin{points}
\item If $\sizeof{F} = q$, then number of points = $q^2$.
\item Lines are solution sets of linear equations ie. $y = mx + c \leftrightarrow \set{(0, c) + \lambda (1, m): \lambda \in F}$, $x = c \leftrightarrow \set{(c, 0) + \lambda (0, 1): \lambda \in F} \implies$ number of lines = $q^2 + q$.
\end{points}
\end{remark}

\begin{result}{Proposition 3.10}
Every line has $q$ points, and every 2 points lie on a unique line ie. $AG(2, F)$ is a 2-design with parameters $(q^2, q, 1)$.
\end{result}

\begin{proof}{Proposition 3.10}
Each line $v + \span{w} = \set{v + \lambda w: \lambda \in F}$ obviously has $q$ points. Now pick $a, b \in F^2$, then $a, b$ lie on $L = \set{a + \lambda (b - a): \lambda \in F}$. Suppose $a, b$ also lie on $L' = v + \span{w}$, then $a = v + \lambda_1 w$, $b = v + \lambda_2 w \implies b - a = (\lambda_2 - \lambda_1)w \implies L' = v + \span{w} = v + \lambda_1 w + \span{w} = a + \span{b - a} = L$.
\end{proof}

In $AG(2, F)$, any 2 lines $L_1, L_2$ meet at 0 or 1 point. If they meet at 0 points, then they are called parallel lines.

\begin{result}{Proposition 3.11}
The $q^2 + q$ lines in $AG(2, F)$ fall into $q + 1$ disjoint sets, each containing $q$ parallel lines.
\end{result}

\begin{proof}{Proposition 3.11}
The $q + 1$ disjoint sets are $\cL_m$ = (set of lines $y = mx + c$ where $c \in F$) for $m \in F$, and $\cL_\infty$ = (set of lines $x = c$ where $c \in F$).
\end{proof}

\begin{remark}
These $q + 1$ sets of lines are called the parallel classes of lines.
\end{remark}

\begin{result}{Proposition 3.12}
Each point in $F^2$ lies in exactly 1 line for each parallel class.
\end{result}

\begin{proof}{Proposition 3.12}
Each parallel class has $q$ disjoint lines, each with $q$ points $\implies$ the result follows easily.
\end{proof}

\subsection{Projective planes}

\begin{definition}
A projective plane is a symmetric 2-design with $\lambda = 1$.
\end{definition}

\begin{remark}
By Theorem 3.7, any 2 blocks of a projective plane meet at 1 point.
\end{remark}

\begin{definition}
Equivalently, a projective plane is a set of points and lines (subsets of points) such that:
\noparskip
\begin{points}
\item any 2 points lie on a unique line,
\item any 2 lines meet at a unique point,
\item $\exists$ 4 points where no 3 points lie on a line.
\end{points}
\end{definition}

\begin{remark}
\begin{points}
\item It follows (not so trivially) that all lines have the same number of points, so a projective plane is indeed a 2-design with $\lambda = 1$. In addition, it is also symmetric.
\item $\exists$ a converse to Theorem 3.7: If $\cB$ is a 2-design with parameters $(v, k, \lambda)$, such that any 2 blocks intersect at exactly $\lambda$ points, then $\cB$ is symmetric.
\end{points}
\end{remark}

\begin{example}
Lines in $AG(2, \Z_3)$ fall into 4 parallel classes $\cL_0, \cL_1, \cL_2, \cL_\infty$. Introduce points $p_0, p_1, p_2, p_\infty$ to each line in $\cL_0, \cL_1, \cL_2, \cL_\infty$ respectively, and add a new line $L_\infty = \set{p_0, p_1, p_2, p_\infty}$, then we have a projective plane.
\end{example}

\begin{result}{Proposition 3.13}
Let $F$ be a finite field, $\sizeof{F} = q$, and start with $AG(2, F)$. Add $q + 1$ new points $p_m\ (m \in F)$ and $p_\infty$ to each line in $\cL_m$ and $\cL_\infty$ respectively, and add a new line $L_\infty = \set{p_m : m \in F} \cup \set{p_\infty}$, then the points $F^2 \cup \set{p_m : m \in F} \cup \set{p_\infty}$ and the new lines form a projective plane.
\end{result}

\begin{proof}{Proposition 3.13}
There are $q^2 + q + 1$ points and $q^2 + q + 1$ lines, and each line has $q + 1$ points $\implies$ the points and lines form a symmetric 2-design. Now pick any 2 distinct points $a, b$. If $a, b \in F^2$, then $a, b$ lie on a unique line in $AG(2, F) \implies a, b$ lie on a unique extended line. If $a \in F^2$ and $b = p_m$ for some $m \in F \cup \set{\infty}$, then by Proposition 3.12, $a$ lies on a unique line $L \in \cL_m \implies$ the unique line containing $a, b$ is $L \cup \set{p_m}$. If $a, b$ are both $p_m$ for some $m \in F \cup \set{\infty}$, then the unique line containing $a, b$ is $L_\infty$. Hence any 2 points lie on a unique line $\implies \lambda = 1 \implies$ the result follows.
\end{proof}

\begin{definition}
This projective plane is called $PG(2, F)$, the projective plane over $F$.
\end{definition}

\begin{remark}
$PG(2, F)$ is a symmetric 2-design with parameters $(q^2 + q + 1, q + 1, 1)$.
\end{remark}

\subsection{Isomorphisms}

\begin{definition}
Let $(X_1, \cB_1)$ and $(X_2, \cB_2)$ be designs. A map $\phi: X_1 \rightarrow X_2$ is an isomorphism of designs if $\phi$ is bijective, and sends the blocks in $\cB_1$ to the blocks in $\cB_2$ bijectively.
\end{definition}

\begin{notation}
If $\exists \phi$, write $(X_1, \cB_1) \cong (X_2, \cB_2)$.
\end{notation}

\begin{example}
Let $X_1 = \Z_7$, $\cB_1$ = 2-design with blocks $B_0 + i$ where $B_0 = \set{0, 1, 3}$, with parameters $(7, 3, 1)$, and $X_2 = \Z_2^3 \setminus \set{0}$, $\cB_2$ = blocks of the form $\set{x, y, x + y}$, which is also a 2-design with parameters $(7, 3, 1)$. We try our luck and construct $\phi: X_1 \rightarrow X_2$, $0 \mapsto 100$, $1 \mapsto 010$, $2 \mapsto 001$, then $\set{0, 1, 3} \mapsto \set{100, 010, 110} \implies 3 \mapsto 110$, $\set{1, 2, 4} \mapsto \set{010, 001, 011} \implies 4 \mapsto 011$, $\cdots$, $5 \mapsto 111$, $6 \mapsto 101 \implies$ the blocks in $\cB_1$ (amazingly) get mapped to the blocks in $\cB_2 \implies (X_1, \cB_1) \cong (X_2, \cB_2)$. In fact, we can map $0 \mapsto x$, $1 \mapsto y$, $2 \mapsto z$ for any $z \notin \set{x, y, x + y}$ to get an isomorphism $\implies$ number of isomorphisms = $7 \times 6 \times 4 = 168$.
\end{example}

\begin{remark}
The set of isomorphisms $\cB \rightarrow \cB$ form a group under composition, called the automorphism group $\aut{\cB}$.
\end{remark}

\subsection{Higher-dimensional geometry}

\begin{definition}
Let $F$ be a finite field with $\sizeof{F} = q$, then $F^n = \set{(x_1, \cdots, x_n): x_i \in F}$.
\end{definition}

\begin{remark}
$F^n$ is an $n$-dimensional vector space over $F$, with $q^n$ vectors.
\end{remark}

\begin{definition}
Let $1 \le m \le n$, then the $q$-binomial coefficient is $\binom{n}{m}_q = \frac{(q^n - 1) \cdots (q^{n - m + 1} - 1)}{(q^m - 1) \cdots (q - 1)}$.
\end{definition}

\begin{example}
\begin{points}
\item $\binom{n}{1}_q = \frac{q^n - 1}{q - 1}$.
\item $\binom{4}{2}_2 = \frac{(2^4 - 1)(2^3 - 1)}{(2^2 - 1)(2 - 1)} = \frac{15 \times 7}{3 \times 1} = 35$.
\item $\binom{n}{m}_1 = \binom{n}{m}$ (consider limits as $q \rightarrow 1$).
\end{points}
\end{example}

\begin{result}{Proposition 3.14}
\begin{points}
\item The number of $m$-dimensional subspaces of $F^n$ is $\binom{n}{m}_q$.
\item For a fixed $v \in F^n \setminus \set{0}$, the number of $m$-dimensional subspaces of $F^n$ containing $v$ is $\binom{n - 1}{m - 1}_q$ if $m > 1$, 1 if $m = 1$.
\item For linearly independent $v, w \in F^n \setminus \set{0}$, the number of $m$-dimensional subspaces of $F^n$ containing $v, w$ is $\binom{n - 2}{m - 2}_q$ if $m > 2$, 1 if $m = 2$.
\end{points}
\end{result}

\begin{proof}{Proposition 3.14}
\begin{points}
\item Let $S(m)$ be the number of $m$-dimensional subspaces of $F^n$, $(w_1, \cdots, w_m)$ be an ordered $m$-tuple of linearly independent vectors in $F^n$, and $W = \span{w_1, \cdots, w_m}$, then the number of pairs $((w_1, \cdots, w_m), W)$ is = ways to choose $(w_1, \cdots, w_m) \times 1$ = ways to choose $w_1\ \times$ ways to choose $w_2 \notin \span{w_1}\ \times \cdots \times$ ways to choose $w_m \notin \span{w_1, \cdots, w_{m - 1}} = (q^n - 1)(q^n - q) \cdots (q^n - q^{m - 1})$. \\
On the other hand, the number of such pairs is also = ways to choose $W\ \times$ ways to choose $(w_1, \cdots, w_m)$ = $S(m)\ \times$ ways to choose $w_1 \in W\ \times$ ways to choose $w_2 \in W \setminus \span{w_1}\ \times \cdots \times$ ways to choose $w_m \in W \setminus \span{w_1, \cdots, w_{m - 1}} = S(m) \times (q^m - 1)(q^m - q) \cdots (q^m - q^{m - 1}) \implies S(m) = \frac{(q^n - 1) \cdots (q^n - q^{m - 1})}{(q^m - 1) \cdots (q^m - q^{m - 1})} = \frac{(q^n - 1) \cdots (q^{n - m + 1} - 1)}{(q^m - 1) \cdots (q - 1)} = \binom{n}{m}_q$.
\item Let $W$ be an $m$-dimensional subspace containing $v$, and $V = \span{v_2, \cdots, v_n}$ where $\set{v, v_2, \cdots, v_n}$ is a basis of $F^n$, then $W \nsubseteq V \implies \dim (W \cap V) = m - 1 \implies W = \span{v} + (W \cap V) \implies$ ways to choose $W$ = (number of $(m - 1)$-dimensional subspaces of $V$) = $\binom{n - 1}{m - 1}_q$ if $m > 1$, 1 if $m = 1$.
\item Similar to (2).
\end{points}
\end{proof}

\begin{result}{Proposition 3.15}
Let $n \ge 2$, $1 \le m \le n - 1$. Define points = vectors $\in F^n$, and blocks = subsets of the form $v + W$ where $v \in F^n$ and $W$ is a $m$-dimensional subspace of $F^n$. Then we have:
\noparskip
\begin{points}
\item a 2-design with parameters $(q^n, q^m, \lambda)$, where $\lambda = \binom{n - 1}{m - 1}_q$ if $m > 1$, 1 if $m = 1$,
\item a 3-design with parameters $(2^n, 2^m, r_3)$ if $F = \Z_2$ and $m \ge 2$, where $r_3 = \binom{n - 2}{m - 2}_2$ if $m > 2$, 1 if $m = 2$.
\end{points}
\end{result}

\begin{proof}{Proposition 3.15}
\begin{points}
\item Note that all blocks $v + W$ have the same size $\sizeof{W} = q^m$. Now pick $v_1, v_2 \in F^n$, $v_1 \ne v_2$, then any block containing $v_1$ is of the form $v_1 + W$, and $v_2 \in v_1 + W \iff v_1 - v_2 \in W \implies$ by Proposition 3.14(2), $\lambda$ = number of blocks containing $v_1, v_2$ = (number of $W$ containing $v_1 - v_2$) = $\binom{n - 1}{m - 1}_q$ if $m > 1$, 1 if $m = 1 \implies$ the result follows.
\item Pick distinct $v_1, v_2, v_3 \in \Z_2^n$, then $v_2, v_3 \in v_1 + W \iff v_2 - v_1, v_3 - v_1 \in W$. Moreover, if $v_2 - v_1$, $v_3 - v_1$ are linearly dependent, then $v_2 - v_1 = c(v_3 - v_1)$ for some $c \in Z_2 \implies c$ = 0 or 1 $\implies v_1 = v_3$ or $v_2 = v_3$ ($\contradiction$) $\implies v_2 - v_1$, $v_3 - v_1$ are linearly independent $\implies$ by Proposition 3.14(3), $r_3$ = number of blocks containing $v_1, v_2, v_3$ = (number of $W$ containing $v_2 - v_1$ and $v_3 - v_1$) = $\binom{n - 2}{m - 2}_q$ if $m > 2$, 1 if $m = 2 \implies$ the result follows.
\end{points}
\end{proof}

\begin{definition}
This design is denoted $AG(n, F)_m$.
\end{definition}

\begin{example}
\begin{points}
\item Let $n = 2$, $m = 1$, then the design is $AG(2, F)$, with blocks of the form $v + \span{w}$ ie. lines in $F^2$.
\item $AG(3, \Z_3)$ is a 2-design with parameters $(27, 3, 1)$.
\item $AG(3, \Z_3)_2$ is a 2-design with parameters $(27, 9, 4)$.
\item $AG(3, \Z_2)_2$ is a 3-design with parameters $(8, 4, 1)$.
\item The codewords of weight 4 in $H'$ form a 3-design isomorphic to $AG(3, \Z_2)_2$.
\end{points}
\end{example}

\subsection{2-designs \& strongly regular graphs}

\begin{definition}
A 2-design is quasi-symmetric if $\exists x, y \in \Z$, $x \ne y$, such that any 2 blocks intersect at either $x$ or $y$ points, and both occur.
\end{definition}

\begin{example}
\begin{points}
\item In $AG(2, F)$, any 2 lines meet at 0 or 1 point $\implies AG(2, F)$ is quasi-symmetric.
\item Consider points = 23 coordinate positions of $G_{23}$, blocks = $B_c$ for $c \in G_{23}$, $\wt{c} = 7$, then we have a 4-design with parameters $(23, 7, 1)$. For $c, d \in G_{23}, \wt{c} = \wt{d} = 7$, $c \ne d$, we have $\wt{c + d} = \wt{c} + \wt{d} - 2[c, d] = 14 - 2[c, d]$ = 8 or 12 $\implies \sizeof{B_c \cap B_d} = [c, d]$ = 3 or 1, and it is easily checked that both occur $\implies$ this design is quasi-symmetric.
\end{points}
\end{example}

\begin{result}{Proposition 3.16}
Let $\Gamma (\ne K_v, K_v^c)$ be a graph with $v$ vertices and adjacency matrix $A$, then TFAE:
\noparskip
\begin{points}
\item $\Gamma$ is strongly regular,
\item $A^2 = \alpha A + \beta I + \gamma J$ for some $\alpha, \beta, \gamma \in \R$.
\end{points}
\end{result}

\begin{proof}{Proposition 3.16}
(1) is true $\implies$ by Proposition 2.6, (2) is also true. \\
(2) is true $\implies$ number of common neighbours of $i, j$ = (row $i$ of $A$) $\cdot$ (row $j$ of $A$) = (row $i$ of $A$) $\cdot$ (column $j$ of $A$) = $(A^2)_{ij}$ = $\beta + \gamma$ if $i = j$, $\alpha + \gamma$ if $i \ne j$ and $i$ is joined to $j$, $\gamma$ if $i \ne j$ and $i$ is not joined to $j \implies \Gamma$ is strongly regular with parameters $(v, \beta + \gamma, \alpha + \gamma, \gamma) \implies$ (1) is also true.
\end{proof}

\begin{result}{Theorem 3.17}
Let $\cB$ be a quasi-symmetric 2-design, such that any 2 blocks intersect at either $x$ or $y$ points. Let $\Gamma(\cB)$ be a graph, with vertices = blocks of $\cB$, and join $B_1, B_2 \in \cB$ iff $\sizeof{B_1 \cap B_2} = x$. Then $\Gamma(\cB)$ is strongly regular.
\end{result}

\begin{proof}{Theorem 3.17}
Let $M$ be the $v \times b$ incidence matrix of $\cB$ and $A$ be the $b \times b$ adjacency matrix of $\Gamma(\cB)$, then $(M^\top M)_{ij}$ = (column $i$ of $M$) $\cdot$ (column $j$ of $M$) = $\sizeof{B_i \cap B_j} = k$ if $i = j$, $x$ if $B_i$ is joined to $B_j$ in $\Gamma(\cB)$, $y$ otherwise $\implies M^\top M = kI_b + xA + y(J_b - A - I_b) = (x - y)A + (k - y)I_b + yJ_b \implies$ since $x \ne y$, $A = rM^\top M + sI_b + tJ_b$ for some $r, s, t \in \R \implies A^2 = r^2 M^\top MM^\top M + s^2 I_b + t^2 J_b^2 + 2rsM^\top M + 2stJ_b + rtM^\top M J_b + rtJ_b M^\top M$. \\
By Proposition 3.3, $MM^\top = \lambda J_v + (r - \lambda) I_v$, $MJ_b = rJ$, $J_v M = kJ$ where $J = v \times b$ matrix consisting of all 1's $\implies
M^\top MM^\top M = M^\top (\lambda J_v + (r - \lambda) I_v) M = (\lambda kJ^\top + (r - \lambda)M^\top)M = \lambda k^2 J_b + (r - \lambda)[(x - y)A + (k - y)I_b + yJ_b] = (r - \lambda)(x - y)A = (r - \lambda)(k - y)I_b + (\lambda k^2 + (r - \lambda)y)J_b$,
$J_b^2 = bJ_b$,
$M^\top M J_b = M^\top (rJ) = r(J^\top M)^\top = rkJ_b$,
and $J_b M^\top M = (MJ_b)^\top M = rJ^\top M = rkJ_b \implies A^2 = \alpha A + \beta I_b + \gamma J_b$ for some $\alpha, \beta, \gamma \in \R \implies$ by Proposition 3.16, $\Gamma(\cB)$ is strongly regular.
\end{proof}

\begin{example}
\begin{points}
\item Let the vertices of $\Gamma$ be lines of $AG(2, F)$, and join $L_1, L_2$ iff $\sizeof{L_1 \cap L_2} = 0$ ie. $L_1$ and $L_2$ are parallel. Then $\Gamma = (K_q)^{q + 1}$, where $q = \sizeof{F}$.
\item Let the vertices of $\Gamma$ be the 253 blocks $B_c$ of $G_{23}$ where $\wt{c} = 7$, and join $B_c, B_d$ iff $\sizeof{B_c \cap B_d} = 3$ ie. $\wt{c + d} = 8$. Then $\Gamma$ is strongly regular, with $k$ = (number of $d$ such that $\wt{c + d} = 8$ for a fixed $c$ with $\wt{c} = 7$).
\end{points}
\end{example}

\end{document}